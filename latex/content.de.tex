%\documentclass[]{article}
% --- language dependent typography stuff ------------------------------

\renewcommand{\say}[1]{„\textit{#1}“}
\setdefaultlanguage[spelling=new]{german}

\renewcommand{\fsNormal}{\fontsize{9.75pt}{11.25pt plus 0.1pt minus 0pt}}
\renewcommand{\fsSmall}{\fontsize{8.5pt}{9.5pt plus 0.1pt minus 0pt}}

\hyphenation{Rol-len-spiel}
\hyphenation{Blatt-hälfte}
\hyphenation{Sofort-Gen-re-Ge-ner-ator}

% --- pdf metadata & stuff ---------------------------------------------

\hypersetup{
	pdftitle={FU: Frei und Universell},
	pdfauthor={Nathan Russell, Tina Trillitzsch,Thorsten Panknin},
	pdfsubject={Ein freies universelles Rollenspielsystem.},
	pdfkeywords={FU, fu, Rollenspiel, System, frei, RSP, RPG}
}

% \renewcommand{\backgroundlayername}{Hintergrund}

% --- fine print ---------------------------------------------------------------

% \renewcommand{\nipajinCopyright}{\copyright\ 2009--2015, Markus Leupold-Löwenthal}
% \renewcommand{\nipajinCredits}{Lektorat: Onno Tasler}
% \renewcommand{\nipajinFineprint}{(Logos und Marken ausgenommen. Bitte nicht einweg-relizensieren.)}

% --- language macros --------------------------------------------------

\newcommand{\zB}{z.\,B.}
\newcommand{\uU}{u.\,U.}

%--------------

% \newcommand{\columnsbegin}{\begin{multicols}{3}}
% \newcommand{\columnsend}{\end{multicols}}
\defaultfontfeatures{Mapping=tex-text,Scale=MatchLowercase}
\setmainfont{Times}
% \setmonofont{Lucida Sans Typewriter}
\usepackage{lmodern}
\usepackage{amssymb,amsmath}
\usepackage{ifxetex,ifluatex}
% \usepackage{fixltx2e} % provides \textsubscript
\ifnum 0\ifxetex 1\fi\ifluatex 1\fi=0 % if pdftex
  \usepackage[T1]{fontenc}
  \usepackage[utf8]{inputenc}
\else % if luatex or xelatex
  \ifxetex
    \usepackage{mathspec}
  \else
    \usepackage{fontspec}
  \fi
  \defaultfontfeatures{Ligatures=TeX,Scale=MatchLowercase}
\fi
% use upquote if available, for straight quotes in verbatim environments
\IfFileExists{upquote.sty}{\usepackage{upquote}}{}
% use microtype if available
\IfFileExists{microtype.sty}{%
\usepackage{microtype}
\UseMicrotypeSet[protrusion]{basicmath} % disable protrusion for tt fonts
}{}
\usepackage{hyperref}
\hypersetup{unicode=true,
            pdfborder={0 0 0},
            breaklinks=true}
\urlstyle{same}  % don't use monospace font for urls
\IfFileExists{parskip.sty}{%
\usepackage{parskip}
}{% else
\setlength{\parindent}{0pt}
\setlength{\parskip}{6pt plus 2pt minus 1pt}
}
\setlength{\emergencystretch}{3em}  % prevent overfull lines
\providecommand{\tightlist}{%
  \setlength{\itemsep}{0pt}\setlength{\parskip}{0pt}}
\setcounter{secnumdepth}{0}
% Redefines (sub)paragraphs to behave more like sections
\ifx\paragraph\undefined\else
\let\oldparagraph\paragraph
\renewcommand{\paragraph}[1]{\oldparagraph{#1}\mbox{}}
\fi
\ifx\subparagraph\undefined\else
\let\oldsubparagraph\subparagraph
\renewcommand{\subparagraph}[1]{\oldsubparagraph{#1}\mbox{}}
\fi

\defaultfontfeatures{Mapping=tex-text,Scale=MatchLowercase}
\setmainfont{Times}


\date{}

\begin{document}

\tableofcontents
\newpage

\section{Die Grundlagen}\label{die-grundlagen}
\begin{multicols}{2}
Der Text von FU geht davon aus, dass wenigstens eine von euch schon von
Rollenspielen gehört hat und eine grobe Vorstellung davon hat, wie sie
funktionieren. Wenn ihr alle keinen Plan habt, was ein Rollenspiel ist,
sucht euch zuerst jemanden, der Bescheid weiß.

\subsection{Was ihr braucht}\label{was-ihr-braucht}
Vor dem Spielen solltet ihr ein paar Sachen zusammensuchen. Hier ist die
Einkaufsliste.

\textbf{Würfel}: Um auszuwerten, wie eure Handlungen ausgehen, verwendet
ihr bei FU normale Würfel mit sechs Seiten. Ihr braucht mindestens
einen, aber besser ist es, wenn jede Spielerin etwa drei Stück hat. In
diesem Regeltext werden sechsseitige Würfel durchgehend als ``W6''
bezeichnet. Wenn davor eine Zahl erscheint (wie bei ``2W6'' oder
``4W6''), bedeutet das, dass ihr so viele Würfel werfen müsst.

\textbf{Stifte und Papier}: Die Spielerinnen müssen sich ein paar
Notizen machen - zu den Einzelheiten ihrer Figur, Wichtigem über ihre
Mission und allerlei anderen Dingen. Die Erzählerin braucht Papier, um
den Überblick über die verschiedenen Elemente der Geschichte zu
behalten.

\textbf{Schmierpapier}: Für Erzählerinnen kann es nützlich sein, einen
kleinen Zettelvorrat zur Hand zu haben, um grobe Schaubilder oder eine
Übersicht über die Ereignisse festzuhalten. Für diesen Zweck eignet sich
auch ein kleines Whiteboard.
% \titlespacing{\section}{0pt}{-60.0em}{-60.0em}
\subsection{Was ihr macht}\label{was-ihr-macht}

Deine Freundinnen und du erzählt gemeinsam eine dramatische und
aufregende Geschichte. Du legst Teile der Spielwelt fest, und alle haben
Gelegenheit, diese Welt zu beeinflussen.

Die meisten Spielerinnen suchen sich wahrscheinlich eine Figur aus, die
eine Hauptrolle spielt - coole Typen, die Dinge anpacken und etwas
bewegen wollen. Jede Figur hat ihre eigenen Stärken, Schwächen und
Ziele, die euch helfen, in ihre Rolle zu schlüpfen. Eine der
Spielerinnen ist die Erzählerin, deren Aufgabe es ist, den anderen zu
helfen, ihre Figuren vor Herausforderungen zu stellen und bei
Regelfragen wenn nötig das letzte Wort zu haben.

Das Spielen besteht aus einer Art Unterhaltung, bei der alle
zusammenwirken, um die Figuren in coole und/oder unterhaltsame
Situationen zu verwickeln. Dann benutzt ihr die Würfel, um zu sehen, was
als nächstes passiert. Manchmal arbeitet ihr alle zusammen, stellt Ideen
in den Raum, schlagt Dinge vor und redet alle durcheinander. Was für ein
Chaos - aber hoffentlich ein produktives Chaos, das Spaß macht. Manchmal
wechselt ihr euch vielleicht auch der Reihe nach ab, beschreibt was eure
Figur jeweils macht und schaut, wie es ausgeht.

\newcommand{\TableResult}{%
	\tabelle{X c}{
		\thead{Wurf} & \thead{Ergebnis} \\
	}{
		6                   & Ja, und \\
		5         & Ja! \\
		4                   & Ja, aber \\
		3                &  Nein, aber \\
		2              & Nein! \\
		1 & Nein, und... \\
	}
}

\subsection{Wie ihr's macht}\label{wie-ihrs-macht}

Wenn du etwas tust und das Ergebnis nicht eindeutig ist, wirf einen W6.
Das Ziel ist es, die ungeraden Zahlen zu vermeiden, also möglichst eine
gerade Zahl zu werfen. Je höher die gerade Zahl, desto besser das
Ergebnis. Wenn du eine ungerade Zahl wirfst, schlägt deine Handlung
entweder fehl, oder sie war nicht ganz so gut wie nötig oder erhofft. Je
niedriger die ungerade Zahl, desto schlechter das Ergebnis. Wenn die
Umgebung, Talente, Ausrüstungsgegenstände oder besondere Fähigkeiten
eine Handlung erleichtern oder erschweren, wirfst du mehrere Würfel und
nimmst das beste oder schlechteste Ergebnis.

\subsection{Vor dem Losspielen}\label{vor-dem-losspielen}

Bevor es richtig losgehen kann, musst du mit deinen Freundinnen
entscheiden, was für eine Art Spiel ihr spielen wollt, wer eure Figuren
sein sollen, und wo eure Geschichte stattfindet. Dieses Wissen hilft den
Spielerinnen, coole Figuren zu erfinden und zeigt der Erzählerin, welche
Rolle sie übernehmen. Vielleicht sind diese Dinge ja schon entschieden,
sei es durch die Erzählerin oder durch ein vorgefertigtes
Spielweltmodul. Falls nicht, überlegt euch gemeinsam etwas, das alle
spannend finden und spielen wollen.

\TableResult
\end{multicols}



\vfill
\begin{center}\rule{1.0\linewidth}{\linethickness}\end{center}
%\columnsbegin
\subsection{Anmerkungen}\label{Anmerkugen}

\begin{multicols}{3}
\subsubsection*{Sprecht vor dem Spielen
miteinander!}\label{sprecht-vor-dem-spielen-miteinander}
\addcontentsline{toc}{subsubsection}{Sprecht vor dem Spielen
miteinander!}

Unterhaltet euch über die Art des Spieles, das ihr spielen wollt, damit
alle von Anfang an wissen, was Sache ist.

Einigt euch über die Grundstimmung, das Thema und eure Erwartungen. Es
ist wichtig zu wissen, ob man für leidenschaftliche Schauspieleinlagen
und blutige Metzeleien Beifall oder eher Buhrufe ernten wird.

Bringt Ideen zur Spielwelt ein, schlagt tolle Bilder, klassische oder
abgefahrene Schlüsselszenen und coole Klischees vor und diskutiert
darüber. So hat jeder eine klare Vorstellung davon, worum es im Spiel
gehen wird.

Diese Diskussion wird auch der Erzählerin deutlich machen, was die
Spielerinnen sich vom Spiel wünschen oder erwarten, zum Beispiel ob sie
sich wie sagenhafte Helden, unterdrückte Außenseiter oder tragische
Antihelden fühlen wollen.

\subsubsection*{Der
Sofort-Genre-Generator}\label{der-sofort-genre-generator}
\addcontentsline{toc}{subsubsection}{Der Sofort-Genre-Generator}

Wollt ihr schnell und ohne Umwege direkt loslegen, schreiben alle
Mitspielerinnen jeweils zwei Genres oder Spielweltideen auf kleine
Zettel und werfen sie in einen Hut. Zieht zwei davon und ihr erhaltet
``Vorstadt-Weltuntergang'', ``Mittelalter-Superhelden'',
``Kampfsport-Gymnasium'', und ähnliches. Diskutiert über Stimmung,
Themen, mögliche Geschichten und Figuren und spielt dann los!

% \vskip-\lastskip     % drop the extra space from enumerate
% \vskip-\prevdepth    % back up by the depth of the last lin
% \columnbreak
\end{multicols}
% \columnsend

\end{document}
