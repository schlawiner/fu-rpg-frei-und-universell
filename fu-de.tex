\documentclass[]{article}
\usepackage{lmodern}
\usepackage{amssymb,amsmath}
\usepackage{ifxetex,ifluatex}
\usepackage{fixltx2e} % provides \textsubscript
\ifnum 0\ifxetex 1\fi\ifluatex 1\fi=0 % if pdftex
  \usepackage[T1]{fontenc}
  \usepackage[utf8]{inputenc}
\else % if luatex or xelatex
  \ifxetex
    \usepackage{mathspec}
  \else
    \usepackage{fontspec}
  \fi
  \defaultfontfeatures{Ligatures=TeX,Scale=MatchLowercase}
\fi
% use upquote if available, for straight quotes in verbatim environments
\IfFileExists{upquote.sty}{\usepackage{upquote}}{}
% use microtype if available
\IfFileExists{microtype.sty}{%
\usepackage{microtype}
\UseMicrotypeSet[protrusion]{basicmath} % disable protrusion for tt fonts
}{}
\usepackage{hyperref}
\hypersetup{unicode=true,
            pdfborder={0 0 0},
            breaklinks=true}
\urlstyle{same}  % don't use monospace font for urls
\usepackage{longtable,booktabs}
\IfFileExists{parskip.sty}{%
\usepackage{parskip}
}{% else
\setlength{\parindent}{0pt}
\setlength{\parskip}{6pt plus 2pt minus 1pt}
}
\setlength{\emergencystretch}{3em}  % prevent overfull lines
\providecommand{\tightlist}{%
  \setlength{\itemsep}{0pt}\setlength{\parskip}{0pt}}
\setcounter{secnumdepth}{0}
% Redefines (sub)paragraphs to behave more like sections
\ifx\paragraph\undefined\else
\let\oldparagraph\paragraph
\renewcommand{\paragraph}[1]{\oldparagraph{#1}\mbox{}}
\fi
\ifx\subparagraph\undefined\else
\let\oldsubparagraph\subparagraph
\renewcommand{\subparagraph}[1]{\oldsubparagraph{#1}\mbox{}}
\fi

\date{}

\begin{document}

\section{Einleitung}\label{einleitung}

FU ist ein Spiel voller Action, Abenteuer und Spaß - ein Rollenspiel mit
großen Ambitionen und von verblüffender Einfachheit.

Mit FU kannst du aufregende Geschichten in jeder nur vorstellbaren
Spielwelt erschaffen, und das mit kleinstmöglichem Aufwand und
Vorbereitung.

FU ist vor allem ein einfaches Spiel. Dieses Buch soll dir dabei helfen,
dich in wenigen Minuten von der Idee ``Diese Spielwelt könnten wir doch
als Rollenspiel spielen!'' zum eigentlichen Spiel zu bringen.

Die Figurenerstellung geht schnell und ist so intuitiv, dass du jede
beliebige Figur spielen kannst; das Spielsystem ist leicht zu erlernen
und ganz einfach zu benutzen.

FU ist universell oder auch ``generisch''. Das bedeutet, dass sich
dieses Gerüst aus Grundregeln nicht auf eine bestimmte
Hintergrundgeschichte oder Spielwelt stützt, sondern nichts vorwegnimmt
oder voraussetzt.

FU ist ein Grundsystem, um das herum du deine eigene Spielwelt und deine
eigenen Geschichten entwickeln kannst.

Dennoch ist FU auf eine bestimmte Art des Spielens ausgelegt: Es eignet
sich besonders für das Improvisieren aus dem Bauch heraus ohne große
Vorbereitung oder detaillierte Planung. Erzählerinnen, die gerne spontan
und unvorbereitet Abenteuer leiten, wird FU sicher gut gefallen, ebenso
Spielerinnen, die keine Lust mehr darauf haben, dass man ihnen sagt, was
alles nicht geht, statt was sie alles tun können.

\subsection{Wie man dieses Buch verwendet}\label{wie-man-dieses-buch-verwendet}

Dieses Buch enthält zwei Informationstypen. Zuerst kommen die
eigentlichen Regeln - sie erklären, wie das Spiel gespielt wird. Die
Regeln sind so gesetzt wie die Texte auf dieser Seite, mit klaren
Überschriften und Unterüberschriften.

FU ist relativ unkompliziert, daher musst du wahrscheinlich nach deinem
ersten Spiel nicht mehr oft in diese Regeln schauen. Trotzdem ist alles,
was du brauchst, klar und eindeutig formatiert.

\begin{quote}
Regelbeispiele sind so formatiert. Überall im Haupttext findest du
solche Blöcke, die dir die Regeln in Aktion zeigen. Auch sie wirst du
nach dem ersten Spiel wahrscheinlich nicht noch einmal brauchen.
\end{quote}

Außerdem gibt es auf vielen Seiten (ganz unten) ``Seitenleisten''. Dort
findest du Regelerläuterungen, genauere Beispiele, Anleitungen, wie man
die Regeln im Spiel konkret einsetzen kann, und Zusatzregeln, die du bei
Bedarf hinzufügen kannst. Die Seitenleisten erkennst du leicht an
\emph{(am Ende gewählte Markierung einsetzen!) einer Linie, die sie vom
Haupttext trennt, und an ihrer dreispaltigen Formatierung.} Dort kannst
du immer dann nachschauen, wenn du Klärung zum ``Wie'' und ``Wann''
einer bestimmten Regel brauchst.

Diese deutsche Übersetzung des englischsprachigen Originals verwendet
der Einfachheit halber durchgehend die weibliche Form - Spielerin,
Erzählerin usw. Es sind aber natürlich immer Spielende aller
Geschlechter eingeschlossen.

\begin{center}\rule{0.5\linewidth}{\linethickness}\end{center}

\section{Die Grundlagen}\label{die-grundlagen}

Der Text von FU geht davon aus, dass wenigstens eine von euch zumindest
schon einmal von Rollenspielen gehört hat und eine grobe Vorstellung
davon hat, wie sie funktionieren. Wenn ihr alle keinen Plan habt, was
ein Rollenspiel ist, sucht euch zuerst jemanden, der Bescheid weiß.

\subsection{Was ihr braucht}\label{was-ihr-braucht}

Vor dem Spielen müsst ihr ein paar Sachen zusammensuchen. Hier ist die
Einkaufsliste.

\textbf{Würfel}: Um auszuwerten, wie eure Handlungen ausgehen, verwendet
ihr bei FU normale Würfel mit sechs Seiten. Ihr braucht mindestens
einen, aber besser ist es, wenn jede Spielerin etwa drei Stück hat. In
diesem Regeltext werden sechsseitige Würfel durchgehend als ``W6''
bezeichnet. Wenn davor eine Zahl erscheint (wie bei ``2W6'' oder
``4W6''), bedeutet das, dass ihr so viele Würfel werfen müsst.

\textbf{Stifte und Papier}: Die Spielerinnen müssen sich ein paar
Notizen machen - zu den Einzelheiten ihrer Figur, Wichtigem über ihre
Mission und allerlei anderen Dingen. Die Erzählerin braucht Papier, um
den Überblick über die verschiedenen Elemente der Geschichte zu
behalten.

\textbf{Schmierpapier}: Für Erzählerinnen kann es nützlich sein, einen
kleinen Zettelvorrat zur Hand zu haben, um grobe Schaubilder oder eine
Übersicht über die Ereignisse festzuhalten. Für diesen Zweck eignet sich
auch ein kleines Whiteboard.

\subsection{Was ihr macht}\label{was-ihr-macht}

Deine Freundinnen und du erzählt gemeinsam eine dramatische und
aufregende Geschichte. Du legst Teile der Spielwelt fest, und alle haben
Gelegenheit, diese Welt zu beeinflussen.

Die meisten Spielerinnen suchen sich wahrscheinlich eine Figur aus, die
eine Hauptrolle spielt - coole Typen, die Dinge anpacken und etwas
bewegen wollen. Jede Figur hat ihre eigenen Stärken, Schwächen und
Ziele, die euch helfen, in ihre Rolle zu schlüpfen. Eine der
Spielerinnen ist die Erzählerin, deren Aufgabe es ist, den anderen dabei
zu helfen, ihre Figuren vor Herausforderungen zu stellen und bei
Regelfragen wenn nötig das letzte Wort zu haben.

Das Spielen besteht aus einer Art Unterhaltung, bei der alle
zusammenwirken, um die Figuren in coole und/oder unterhaltsame
Situationen zu verwickeln. Dann benutzt ihr die Würfel, um zu sehen, was
als nächstes passiert. Manchmal arbeitet ihr alle zusammen, stellt Ideen
in den Raum, schlagt Dinge vor und redet alle durcheinander. Was für ein
Chaos - aber hoffentlich ein produktives Chaos, das Spaß macht. Manchmal
wechselt ihr euch vielleicht auch der Reihe nach ab, beschreibt was eure
Figur jeweils macht und schaut, wie es ausgeht.

\subsection{Wie ihr's macht}\label{wie-ihrs-macht}

Wenn du etwas tust und das Ergebnis ist nicht eindeutig, wirf einen W6.
Das Ziel ist es, die ungeraden Zahlen zu vermeiden, also möglichst eine
gerade Zahl zu werfen. Je höher die gerade Zahl, desto besser das
Ergebnis. Wenn du eine ungerade Zahl wirfst, schlägt deine Handlung
entweder fehl, oder sie war nicht ganz so gut wie nötig oder erhofft. Je
niedriger die ungerade Zahl, desto schlechter das Ergebnis. Wenn die
Umgebung, Talente, Ausrüstungsgegenstände oder besondere Fähigkeiten
eine Handlung erleichtern oder erschweren, wirfst du mehrere Würfel und
nimmst das beste oder schlechteste Ergebnis.

\subsection{Vor dem Losspielen}\label{vor-dem-losspielen}

Bevor es richtig losgehen kann, musst du mit deinen Freundinnen
entscheiden, was für eine Art Spiel ihr spielen wollt, wer eure Figuren
sein sollen, und wo eure Geschichte stattfindet. Dieses Wissen hilft den
Spielerinnen, coole Figuren zu erfinden und zeigt der Erzählerin, welche
Rolle sie in der Welt übernehmen wollen. Vielleicht sind diese Dinge ja
schon entschieden, sei es durch die Erzählerin oder durch ein
vorgefertigtes Spielweltmodul. Falls nicht, überlegt euch gemeinsam
etwas, das alle spannend finden und spielen wollen.

\begin{center}\rule{0.5\linewidth}{\linethickness}\end{center}

\columnsbegin

\subsubsection*{Sprecht vor dem Spielen
miteinander!}\label{sprecht-vor-dem-spielen-miteinander}
\addcontentsline{toc}{subsubsection}{Sprecht vor dem Spielen
miteinander!}

Unterhaltet euch über die Art des Spieles, das ihr spielen wollt, damit
alle von Anfang an wissen, was Sache ist.

Einigt euch über die Grundstimmung, das Thema und eure Erwartungen. Es
ist zum Beispiel wichtig zu wissen, ob man für leidenschaftliche
Schauspieleinlagen und blutige Metzeleien Beifall oder eher Buhrufe
ernten wird.

Bringt Ideen zur Spielwelt ein, schlagt tolle Bilder, klassische oder
abgefahrene Schlüsselszenen und coole Klischees vor und diskutiert
darüber. So hat jeder eine klare Vorstellung davon, worum es im Spiel
gehen wird.

Diese Diskussion wird auch der Erzählerin deutlich machen, was die
Spielerinnen sich vom Spiel wünschen oder erwarten, zum Beispiel ob sie
sich wie sagenhafte Helden, unterdrückte Außenseiter oder tragische
Antihelden fühlen wollen.

\subsubsection*{Der
Sofort-Genre-Generator}\label{der-sofort-genre-generator}
\addcontentsline{toc}{subsubsection}{Der Sofort-Genre-Generator}

Wollt ihr schnell und ohne Umwege direkt loslegen, schreiben alle
Mitspielerinnen jeweils zwei Genres oder Spielweltideen auf kleine
Zettel und werfen sie in einen Hut. Zieht zwei davon und ihr erhaltet
``Vorstadt-Weltuntergang'', ``Mittelalter-Superhelden'',
``Kampfsport-Gymnasium'', und ähnliches. Diskutiert über Stimmung,
Themen, mögliche Geschichten und Figuren und spielt dann los!

\columnsend

\begin{center}\rule{0.5\linewidth}{\linethickness}\end{center}

\section{Die Figuren}\label{die-figuren}

Bei FU ist deine Figur dein ``Alter Ego'', dein zweites Ich. Im Verlauf
der Geschichte beschreibst du, was die Figur macht und wie sie auf
Situationen und auf Personen reagiert.

\subsection{Konzept}\label{konzept}

Sobald du weisst, in welcher Spielwelt und unter welchen Voraussetzungen
euer Abenteuer stattfindet, kannst du dir langsam Gedanken über deine
Figur machen - über ihr ``Konzept''. Dies ist der Kern deiner Figur und
beschreibt, was sie ausmacht und wer sie ist. Ein Konzept lässt sich in
wenigen Wörtern oder einem kurzen Satz zusammenfassen.\\
Das Konzept kann den Hintergrund oder den Beruf deiner Figur
beschreiben, wie ``Paranormaler Ermittler'' oder ``Wunderkind''. Oder
vielleicht gibt es Aufschluss über ihre Persönlichkeit, wie ``Edler
Wilder'' oder ``Verrückter Professor''.\\
Das Konzept deiner Figur sollte natürlich zur Spielwelt, zum Hintergrund
und zur Art der Abenteuer passen, die ihr spielt. Ein ``Gewiefter
Bulle'' passt vielleicht nicht ganz ins mittalterliche England, aber ein
``Weltgewandter Sheriff'' könnte hier genau das Richtige sein. Lass dich
beim Konzept deiner Figur von der Spielwelt inspirieren!

\begin{quote}
Im Verlauf dieses Kapitels folgen wir David und Nina bei der Erstellung
ihrer Figuren. Zusammen mit Tim, dem Erzähler, haben sie sich
entschlossen, ein Spiel zu spielen, das auf den amerikanischen
``Pulp''-Magazinen - einer Art Groschenromanen - der 1930er Jahre
basiert. Es soll eine rasante und eher skurrile Geschichte werden. David
lässt sich von seinen Lieblingsfilmen inspirieren und nimmt als Konzept
für seine Figur ``Tollkühner Entdecker''. Nina möchte jemand
Ungewöhnlichen spielen, und nachdem sie ihre Ideen mit Tim und David
diskutiert hat, entscheidet sie sich für eine ``Außerirdische
Botschafterin''.
\end{quote}

\subsection{Beschreiber}\label{beschreiber}

Beschreiber sind Adjektive oder kurze Sätze, mit denen ihr die
Fähigkeiten und Schwachstellen eurer Figuren kennzeichnnet; Dinge, die
ihnen das Leben erleichtern oder erschweren, und die sie letztlich zu
dem machen, was sie sind. Sie zeigen, was eine Figur gut kann, was ihre
körperlichen und geistigen Eigenschaften sind, und beschreiben ihre
Talente und eventuellen Schattenseiten. Einerseits dienen sie zur
Orientierung beim Hineinversetzen in deine Rolle und andererseits als
Modifikator beim Würfeln.

Jede Figur hat vier Beschreiber: Körper, Geist, Stärke und Schwäche.
Beschreibe die markantesten Merkmale deiner Figur mit einem Adjektiv
oder einem kurzem Satz.

\begin{quote}
David überlegt sich, welche Situationen sein Tollkühner Entdecker schon
so erlebt haben könnte und entscheidet sich für die folgenden
Beschreiber:

\textbf{Körper} : Geschickt\\
\textbf{Geist}: Besonnen\\
\textbf{Stärke}: Bullenpeitsche einsetzen\\
\textbf{Schwäche}: Höhenangst
\end{quote}

\begin{quote}
Nina lässt sich von klassischen Science-Fiction-Romanheften inspirieren
und entscheidet sich für die folgenden Beschreiber für ihre Figur, die
Außerirdische Botschafterin:

\textbf{Körper}: Zerbrechlich\\
\textbf{Geist}: Aufmerksam\\
\textbf{Stärke}: Gedankenlesen\\
\textbf{Schwäche}: Außerirdische Sichtweise
\end{quote}

\subsubsection{Beispiele für
Beschreiber}\label{beispiele-fuxfcr-beschreiber}

Dies soll auf keinen Fall eine vollständige Auflistung aller möglichen
Beschreiber sein, sondern euch nur ein paar Ideen an die Hand geben.
Jeder Beschreiber ist im Anhang noch einmal genauer erläutert.

\begin{quote}
\textbf{Körper}: Beweglich, Beidhändig, Blondine, Pelzig,
Selbstbräuner-Orange, Gutaussehend, Gewaltig, Übergewichtig, Schwache
Konstitution, Flink, Rasiermesserscharfe Klauen, Klein, Langsam, Stark,
Hochgewachsen, Dünn, Hässlich, Vitalität, Schwächlich, Schwimmhäute an
den Füßen.

\textbf{Geist}: Zerstreut, Belesen, Computerfreak, Dämlich, Einfühlsam,
Konzentriert, Querdenkerin, Mathematikerin, Aufmerksam, Rednerin,
Langsam, Taktikerin, Ungebildet, Weise, Geistreich.

\textbf{Stärke}: Akrobatik, Okkultes Geheimwissen, Mut, Fahren, Fechten,
Gutes Gedächtnis, Jagen, Scharfe Augen, Sprachwissenschaft, Magie,
Medizin, Fieser Biss, Reich, Ringkampf.

\textbf{Schwäche}: Blind, Mutig, Ungeschickt, Gierig, Unmenschliches
Aussehen, Fehlendes Bein, Alt, Arm, Sieht schlecht, Primitiv, Intensiver
Geruch, Gesucht, Jung.
\end{quote}

\begin{center}\rule{0.5\linewidth}{\linethickness}\end{center}

\columnsbegin

\paragraph*{Klischees sind deine
Freunde!}\label{klischees-sind-deine-freunde}
\addcontentsline{toc}{paragraph}{Klischees sind deine Freunde!}

Es ist völlig in Ordnung, wenn du dich bei deinem Konzept auf Klischees
stützt. Klischees umfassen eine Unzahl von Vorstellungen und Konzepten,
mit denen man direkt vertraut ist. Wenn jemand sagt, seine Figur sei ein
Barbar, hat man automatisch einen muskelbepackten unzivilisierten Klotz
vor Augen. Klischees sind eine Art Abkürzung beim Beschreiben deiner
Figur.

\paragraph*{Beschreiber wählen}\label{beschreiber-wuxe4hlen}
\addcontentsline{toc}{paragraph}{Beschreiber wählen}

Gib dir Mühe, fantasievoll und ehrlich zu sein. Denke an dein Konzept
und die Spielwelt. Einer Figur die Schwäche \textbf{``kann nicht
schwimmen''} zu geben, wenn du weißt, dass alle Geschichten in der Wüste
stattfinden, ist eher unsportlich. Lass deine Fantasie spielen, behalte
dein Konzept im Hinterkopf und besprich deine Ideen mit der Erzählerin.
Es gibt endlos viele Möglichkeiten.

\paragraph*{Beschreiber sind klar und
eindeutig}\label{beschreiber-sind-klar-und-eindeutig}
\addcontentsline{toc}{paragraph}{Beschreiber sind klar und eindeutig}

Beschreiber sind gut, wenn alle am Tisch sie verstehen. Wenn du oder
eine Mitspielerin einen Beschreiber nicht eindeutig findet, redet
darüber. Vielleicht muss er umgeschrieben werden, vielleicht aber auch
nicht. Spätestens dann, wenn er schlussendlich zum Einsatz kommt,
sollten sich alle über seine Bedeutung einig sein.

\paragraph*{Beschreiber gehören fest zu einer
Figur}\label{beschreiber-gehuxf6ren-fest-zu-einer-figur}
\addcontentsline{toc}{paragraph}{Beschreiber gehören fest zu einer
Figur}

Jeder Beschreiber ist ein wesentlicher Teil der Figur, der fest mit ihr
verbunden ist. Normalerweise können Beschreiber ihr also auch nicht
weggenommen oder entfernt werden oder verloren gehen (allerdings können
sie unter Umständen vergessen oder eingeschränkt sein). Mach deine
Beschreiber also nicht zu Gegenständen oder Geräten.

\paragraph*{Die Reichweite von Beschreibern ist
begrenzt}\label{die-reichweite-von-beschreibern-ist-begrenzt}
\addcontentsline{toc}{paragraph}{Die Reichweite von Beschreibern ist
begrenzt}

Jeder Beschreiber sollte ein oder zwei konkrete, offensichtliche
Anwendungsgebiete haben. Er kann natürlich auch in einer ganzen Reihe
von anderen, nicht vorhersehbaren Umständen anwendbar sein, doch das
ergibt sich erst im Laufe des Spiels. \textbf{Medizin} ist
beispielsweise besser als \textbf{Arzt}, denn Letzteres kann auch in
allerlei nichtmedizinischen Situationen eingesetzt werden (\emph{``Als
Arzt bin ich gebildet und ziemlich wohlhabend \ldots{}''}).

\paragraph*{Was für ein Beschreiber ist
das?}\label{was-fuxfcr-ein-beschreiber-ist-das}
\addcontentsline{toc}{paragraph}{Was für ein Beschreiber ist das?}

Manche Beschreiber passen problemlos in mehr als eine Kategorie -
\textbf{Gutes Gedächtnis} könnte zum Beispiel entweder ein Geist- oder
ein Stärke-Beschreiber sein. \textbf{Hässlich} kann ein Schwäche- oder
ein Körper-Beschreiber sein. Das macht nichts. Ob ein Beschreiber für
die Geschichten, die ihr erzählen wollt passt oder nicht hängt von dir,
der Erzählerin und von deinen Mitspielerinnen ab. Entscheidet selbst.

\paragraph*{Was sind gute Stärken und
Schwächen?}\label{was-sind-gute-stuxe4rken-und-schwuxe4chen}
\addcontentsline{toc}{paragraph}{Was sind gute Stärken und Schwächen?}

Dinge, die in anderen Spielen ``Fertigkeiten'' oder ``Talente'' heißen,
eignen sich wunderbar als Stärken. Gute Beispiele sind
\textbf{``Schwertkampf''}, \textbf{``unter Wasser atmen''} und
\textbf{``gewinnendes Lächeln''}. Die besten Schwächen sind
Persönlichkeitsmerkmale oder körperliche Defizite. Sachen wie
\textbf{``riecht immer schlecht''} oder \textbf{``taub''} sind bessere
Schwächen als \textbf{``kann nicht Fahren''} oder \textbf{``kann kein
Französisch''}. Es gibt natürlich immer auch Ausnahmen. Wenn die Figuren
britische Spione zu Zeiten der Napoleonischen Kriege sind, wäre es ein
echter Nachteil, kein Französisch zu sprechen. Genauso kann die Schwäche
\textbf{``kann nicht schwimmen''} in einem Spiel mit Piraten und
Seekämpfen zu einer echten Attraktion werden. Suche immer nach
Möglichkeiten, deine Figur herauszufordern, dem Spiel Würze zu verleihen
und Hindernisse einzubringen, die überwunden werden müssen.

\paragraph*{Soll ich mich
spezialisieren?}\label{soll-ich-mich-spezialisieren}
\addcontentsline{toc}{paragraph}{Soll ich mich spezialisieren?}

Du kannst deine Beschreiber eng auf ein einziges Konzept konzentrieren
und dich so ``spezialisieren''. Einem richtig furchteinflößenden Krieger
kannst du Körper: \textbf{Stark}, Geist: \textbf{Taktiker}, Stärke:
\textbf{Nahkampf} und Schwäche: \textbf{Leichtsinnig} geben. Es wäre
nicht schwer, fast alle dieser Beschreiber ins Spiel zu bringen, wenn
ihr in einem Kampf geratet. Jedoch hat die Figur so kaum Tiefe - in
Situationen ohne Kampf wirst du blöd dastehen. Stattdessen könntest du
ein paar Beschreiber durch vielseitigere ersetzen - Körper:
\textbf{Zäh}, Geist: \textbf{Konzentriert}, Stärke: \textbf{Nahkampf}
und Schwäche: \textbf{Leichtsinnig}. Beschreibe den Körper deiner Figur
als \textbf{Gewaltig} und ihre Stärke als \textbf{Stark}, wenn du einen
Ringer, Bodybuilder oder wütenden grünen Superhelden willst. Gib deiner
genialen Wissenschaftlerin Geist: \textbf{Belesen} und Stärke:
\textbf{Raketentechnik}. Bedenke aber: Je stärker du dich
spezialisierst, desto mehr musst du dich in Szenen anstrengen, die
nichts mit deinen Begabungen zu tun haben. Einige Spielerinnen empfinden
das als willkommene Herausforderung.

\paragraph*{Beschreiber-Alternativen}\label{beschreiber-alternativen}
\addcontentsline{toc}{paragraph}{Beschreiber-Alternativen}

Die vier Standard-Beschreiber (Körper, Geist, Stärke und Schwäche) sind
nicht die einzige Möglichkeit, deine Figuren zu definieren. Ändert oder
ersetzt sie, um sie an eure Spielwelt und eure Geschichten anzupassen.
In einem Spiel über Riesen-Kampfroboter könntet ihr Körper und Geist zum
Beispiel durch Chassis und Pilot ersetzen. In einem Spiel, in dem jeder
unterschiedliche Wer-Wesen spielt, könnte ein Beschreiber zu
``Tierform'' werden. In solchen Fällen müssen ein paar der zuvor
erwähnten Ratschläge ebenfalls angepasst werden; Tierform:
\textbf{Jaguar} ist viel allgemeiner als die bisher angesprochenen
Beschreiber, kann aber in diesem Fall genau passen, um den Ansatz eures
Spiels abzubilden. Seid kreativ und macht bei Bedarf eigene Änderungen.

\columnsend
---

\subsection{Ausrüstung}\label{ausruxfcstung}

Ausrüstung beschreibt das Zubehör, die coolen Geräte und die Waffen, die
deine Figur in ihren Abenteuern benutzt. Alle Figuren besitzen die für
ihr Konzept üblichen Kleider und Gegenstände. Ausrüstung dagegen umfasst
die wirklich wichtigen und coolen Sachen, die sie bei sich tragen.

Ausrüstung kann, genau wie Beschreiber, Würfelwürfe modifizieren. Sie
kann aus allem Möglichen bestehen - von Schusswaffen über Messer bis hin
zu ausgefallener Garderobe, Kreditkarten, einem Pferd, einem
Mobiltelefon oder sogar einem Raumschiff. Wie bei Beschreibern ist die
Ausrüstung deiner Figur abhängig von deinem Konzept, der Spielwelt und
den Geschichten, die ihr erzählen wollt. \emph{Anders als Beschreiber}
besteht ein Ausrüstungsgegenstand immer aus einem Adjektiv und einem
Substantiv (oder zusammengesetzen Substantiv): Rostiges Schwert, Langes
Seil, Schnelles Pferd, Scharfschützengewehr, Papas Camaro, Schwere
Rüstung.

Deine Figur hat genau zwei Ausrüstungsgegenstände, such dir also zwei
Teile aus.

\begin{quote}
David entscheidet, dass sein Tollkühner Entdecker eine \textbf{Stabile
Bullenpeitsche} dabei hat, da er damit recht geschickt ist, und eine
\textbf{Abgetragene Lederjacke}, mit der er sich vor der Kälte und
leichten Kratzern und Stürzen schützen kann.\\
Nina denkt erst kurz über ihre Ausserirdische Botschafterin nach und
entscheidet sich dann für \textbf{Offizielle Dokumente}, die ihren
politischen Status belegen, und für \textbf{Exotische Gewänder}, die
Eindruck machen und Ehrfurcht einflößen sollen.
\end{quote}

\subsection{Beispiel-Ausrüstung}\label{beispiel-ausruxfcstung}

Wie bei den Beispielen für Beschreiber gilt auch hier, dass diese Liste
auf keinen Fall alles abdeckt, was möglich ist. Sie stellt lediglich die
Spitze des Eisbergs dessen dar, was deine Figur besitzen könnte.

\textbf{Kleider}: Designer-Jeans, Hautenges schwarzes Kleid (lassen wir
``schwarzes Kleid'' mal als Substantiv gelten), Dreckige Unterhosen,
Abgetragene Jacke, Supermoderner Raumanzug, Seidener Hausrock,
Verbeulter Filzhut, Hohe Puderperücke.

\textbf{Waffen}: Rostiger Säbel, Schwere Axt, Meines Vaters
Dienstrevolver, Verborgener Dolch, Experimenteller Flammenwerfer,
Falscher Revolver, Zuverlässige AK-47, Gummihammer.

\textbf{Fortbewegung}: Treues Pferd, Verbeulter Buick, Gepanzerte
Luxuskarosse, Quietschende Clownsschuhe, Aufgemotzter Straßenrennwagen,
Mädchenfahrrad, Unzuverlässiges Sportcoupé, Klappriger Karren, Schnelles
Motorrad, Omas Flitzer.

\textbf{Sonstiges}: Riesiger Rucksack, Kleine Flagge, Schwere
Rollenspielbücher, Treuer Hund, Verbeulte Bratpfanne, Nassgewordenens
Notizbuch, Uraltes Zauberbuch, Mein Lieblingskieselstein.

\begin{center}\rule{0.5\linewidth}{\linethickness}\end{center}

\columnsbegin

\subsubsection{Ausrüstung auswählen}\label{ausruxfcstung-auswuxe4hlen}

Die Gegenstände, die du als Ausrüstung wählst, sollten ein Markenzeichen
für deine Figur sein. Denk zum Beispiel an Batmans \textbf{Düsteren
Fledermausanzug}, die \textbf{Instabilen Protonen-Packs} der
Ghostbusters, James Bonds \textbf{Zuverlässige Berretta} oder Zorros
\textbf{Blitzende Klinge}.

Jedes Ausrüstungsteil sollte etwas zur Charakterisierung deiner Figur
beitragen, entweder zu ihrem Hintergrund, ihrer Persönlichkeit oder
ihren Zielen. Es sollte etwas über deine Figur oder ihre Taten aussagen.

\subsubsection{Ausrüstung ist
Zubehör}\label{ausruxfcstung-ist-zubehuxf6r}

Ausrüstung ist nie fest zu einer einer Figur - sie kann fallengelassen
werden, verloren oder kaputt gehen oder gestohlen werden. Ein
Kybernetischer Arm ist keine Ausrüstung, ein \textbf{Gepanzerter
Roboterhandschuh} dagegen schon.

\subsubsection{Ausrüstung beschreiben}\label{ausruxfcstung-beschreiben}

Wenn du deine Ausrüstung beschreibst, lass das dazugehörige Adjektiv
etwas Nützliches und/oder Interessantes über den Gegenstand aussagen.
Ein \textbf{Langer Dolch} ist zwar ganz gut, aber ein
\textbf{Zerbrochener Dolch} ist noch besser!

Die Beschreibung deiner Ausrüstung sollte, wie Beschreiber auch, klar
und eindeutig sein - es dürfen keine Zweifel darüber bestehen, wofür sie
gut sind oder was ihre Haupteigenschaft ist.

\subsubsection{Nur ein einziges
Adjektiv}\label{nur-ein-einziges-adjektiv}

Die Beschreibung jedes Ausrüstungsgegenstandes sollte sich auf genau ein
Adjektiv beschränken - nicht mehr und nicht weniger. Ein
\textbf{Scharfer Säbel} ist gut, ein \textbf{Magisches Schwert} auch,
aber ein \textbf{Scharfer Magischer Säbel} ist nicht erlaubt.

\subsubsection{Konkrete Substantive}\label{konkrete-substantive}

Wähle konkrete, aussagekräftige Substantive. Säbel ist besser als
Schwert und Baseballkappe ist besser als Mütze. Du darfst auch mehr als
ein Substantiv benutzen, aber beschränke dich auf so wenige Wörter wie
möglich.

\subsubsection{Sprich über deine
Ausrüstung}\label{sprich-uxfcber-deine-ausruxfcstung}

Diskutiere deine Ausrüstung mit der Gruppe. Allen muss klar sein, was du
beschreibst - wofür die Ausrüstung nützlich ist und wobei sie eher
hinderlich sein kann.

Ausrüstung ist nie von Haus aus ``gut'' oder ``schlecht''. Ob der Besitz
eines Gegenstands nützlich ist oder nicht, hängt davon ab, was du damit
machst und in welcher Situation du bist.

\subsubsection{Gute Ausrüstung und schlechte
Ausrüstung}\label{gute-ausruxfcstung-und-schlechte-ausruxfcstung}

Wenn ihr wollt, könnt ihr festlegen, dass der eine Ausrüstungsgegenstand
ein ``gutes'' Adjektiv, und der andere ein ``schlechtes'' haben muss.
Vielleicht hast du eine \textbf{``Warme Jacke''} und eine \textbf{``Alte
Kanone''} oder ein \textbf{``Schnelles Motorrad''} und eine
\textbf{``Überzogene Kreditkarte''}.

\subsubsection{Kram, der keine Ausrüstung
ist}\label{kram-der-keine-ausruxfcstung-ist}

Alle Gegenstände oder Geräte, die nicht als Ausrüstung auf dem
Figurenbogen stehen, sind Requisiten. Requisiten haben keine Auswirkung
auf die Erfolgsaussichten einer Handlung - sie sind lediglich
kosmetischer Natur, quasi Schaufensterdekoration. Du kannst aber die
Ausrüstung mit anderen Figuren tauschen, sie ihnen stehlen oder vom
Boden aufheben und sie dann selbst benutzen.

\columnsend

\begin{center}\rule{0.5\linewidth}{\linethickness}\end{center}

\subsection{Beschreibung}\label{beschreibung}

Inzwischen hast du bestimmt schon eine ganz gute Vorstellung von deiner
Figur. Jetzt ist es soweit, die restlichen Einzelheiten zu erfinden.
Hier beschreibst du Aussehen und Persönlichkeit deiner Figur, ihre
Vergangenheit, ihre Ziele, Freunde und Feinde und alle anderen
Einzelheiten, die du wichtig oder interessant findest.

\begin{quote}
David macht sich ein paar Notizen zu seinem Tollkühnen Entdecker:

Tennessee Smith ist ein friedfertiger Geschichtsprofessor, der in seiner
Freizeit allerdings oft in die Wildnis reist, um verschollene Artefakte
und Schätze zu suchen. Er ist ein kerniger, attraktiver Typ und scheint
immer die Ruhe zu bewahren, egal in welcher Gefahr er sich auch
befindet. Und in Gefahr gerät er ziemlich oft!
\end{quote}

\begin{quote}
Ninas Beschreibung ihrer Ausserirdischen Botschafterin ist
folgendermaßen:

Lumina ist eine wichtige Diplomatin vom Planeten Jupiter, welcher vom
Schreckensfürst Kang regiert wird. Wie alle ihrer Art hat sie lila Haut,
einen haarlosen Kopf und feine Gesichtszüge. Lumina arbeitet für die
Jupiter-Untergrundbewegung und versucht, Schreckensfürst Kang zu
stürzen.
\end{quote}

\subsection{Motive}\label{motive}

Jede Figur hat einen Antrieb - ein Ziel, das sie anstrebt. Das muss
nichts Weltbewegendes sein (aber es kann!), sollte der Figur aber
trotzdem wichtig sein. Stelle deiner Figur die folgenden Fragen:

\textbf{Was willst du erreichen?} Was wünschst du dir, was treibt dich
an?

\textbf{Was hindert dich daran?} Welches Hindernis, welche Gegner stehen
zwischen dir und dem, was du willst?

\textbf{Was bist du bereit du tun?} Was ist dein nächster Schritt auf
dem Weg zu deinem Ziel? Was würdest du tun, um es zu schaffen?

\begin{quote}
Tennessee Smith ist auf der Suche nach dem Götzen des Tot, einem
Artefakt, das er schon seit Jahrzenten heiß begehrt. Sein Rivale Giles
Fishburne ist ebenfalls hinter dem Götzen her und scheint ihm fast immer
einen Schritt voraus zu sein. Dieses Mal will Tennessee Smith es
unbedingt schaffen und ist zu allem bereit, um an den Götzen zu gelangen
- doch töten würde er dafür niemals.
\end{quote}

\begin{quote}
Lumina strebt nach Freiheit für alle Völker des Jupiter. Schreckensfürst
Kang regiert den Planeten mit eiserner Hand und seine Agenten halten
ständig Ausschau nach Abtrünnigen. Lumina ist bereit, ihr eigenes Leben
aufs Spiel zu setzen, um ihren Traum zu verwirklichen.
\end{quote}

\subsection{Beziehungen}\label{beziehungen}

Wähle mindestens eine andere Figur, die in der Geschichte mitspielt und
schreibe eine kurze Aussage zu deiner Beziehung mit ihr auf. Sie sollte
klar und eindeutig sein und der Vergangenheit beider Figuren etwas
vertiefen. Zum Beispiel: \textbf{Alte Saufkumpane}, \textbf{Mit
derselben Frau liiert}, \textbf{Zusammen im Krieg gekämpft}, \textbf{Vom
selben Meister ausgebildet}.

\begin{quote}
David schreibt: Tennessee Smith und Lumina haben sich in Harvard
getroffen, wo Lumina gerade eine Vorlesung über die antiken Kulturen des
Jupiter hielt. Nina ist damit zufrieden und fügt noch hinzu, dass beide
sich zueinander hingezogen fühlten, aber Lumina für romantische
Tändeleien zu sehr auf ihre Mission konzentriert ist.
\end{quote}

\begin{center}\rule{0.5\linewidth}{\linethickness}\end{center}

\columnsbegin

\subsubsection{Jetzt dreht sich alles ums
Rollenspiel}\label{jetzt-dreht-sich-alles-ums-rollenspiel}

Deine Beschreibung, Motive und Beziehungen sollen dir dabei helfen,
deine Figur zu entwickeln und sie in die Welt deiner Geschichten und
Abenteuer einzubinden. Die Erzählerin gibt dir vielleicht gelegentlich
einen Bonus auf Würfelwürfe für Dinge, die sich aus diesen drei Dingen
ableiten lassen, aber das ist keinesfalls ein Muss.

Nutze deine Motive und Beziehungen als Richtlinie dafür, wie du deine
Rolle spielst und wie deine Figur auf die der anderen Spielerinnen und
auf ihre Umwelt reagiert.

\subsubsection{Langfristige oder kurzfristige
Ziele?}\label{langfristige-oder-kurzfristige-ziele}

Wenn du dein Motiv aussuchst, kannst du frei zwischen lang- und
kurzfristigen Zielen wählen. Wenn ihr nur ein Spiel mit einer einzigen
Sitzung spielt, ist es am besten, etwas zu nehmen, das eine sofortige
und direkte Wirkung auf die Geschichte hat. Wenn ihr eine ganze Reihe
von Spielen plant, kannst du dir aber für deine Figur durchaus ein Ziel
ausdenken, dessen Erreichen länger braucht.

\subsubsection{Wie viele Beziehungen?}\label{wie-viele-beziehungen}

Zwei Beziehungen sind ein guter Anfang. Such dir zwei verschiedene
Figuren aus und entscheide, woher sie sich kennen. Du kannst gemeinsam
mit einer Mitspielerin die Beziehung zwischen euren Figuren festlegen,
oder ihr nehmt getrennte Beziehungen, die nichts miteinander zu tun
haben - oder sogar ``gegenläufige'' Beziehungen (zum Beispiel:
\textbf{``Verliebt in''/``Angewidert von''}).

\subsubsection{Seid nicht eure eigenen
Feinde}\label{seid-nicht-eure-eigenen-feinde}

Entwerft keine Motive oder Beziehungen, bei denen sich die Figuren
dauernd in den Haaren liegen. Es macht nichts, wenn Figuren nicht immer
einer Meinung sind (das kann sogar Spaß machen!), aber sie sollten sich
nicht hassen. Die Figuren müssen schließlich zusammenhalten, um Feinde
bezwingen, das Geheimnis lösen oder den Auftrag erfüllen zu können.

\columnsend

\begin{center}\rule{0.5\linewidth}{\linethickness}\end{center}

\newpage

\subsection{Figurenerstellung
kurzgefasst}\label{figurenerstellung-kurzgefasst}

\subsubsection*{1. Konzept}\label{konzept-1}
\addcontentsline{toc}{subsubsection}{1. Konzept}

Wer ist deine Figur? Was ist ihr Grundkonzept? Klischees und Archetypen
sind ausdrücklich erlaubt!

\subsubsection*{2. Beschreiber}\label{beschreiber-1}
\addcontentsline{toc}{subsubsection}{2. Beschreiber}

Lege die vier wichtigsten oder interessantesten Merkmale deiner Figur
fest: jeweils eines für Körper, Geist, Stärke und Schwäche. Jeder dieser
Beschreiber sollte kurz, ausdrucksstark, eindeutig, der Figur fest
zugehörig und begrenzt sein.

\subsubsection*{3. Ausrüstung}\label{ausruxfcstung-1}
\addcontentsline{toc}{subsubsection}{3. Ausrüstung}

Welches coole Zeug trägt deine Figur als Markenzeichen bei sich? Wähle
zwei Ausrüstungsgegenstände und beschreibe sie jeweils mit einem
Adjektiv und einem Substantiv (z.B. Scharfer Säbel, Schnelles Motorrad)

\subsubsection*{4. Beschreibung}\label{beschreibung-1}
\addcontentsline{toc}{subsubsection}{4. Beschreibung}

Wie sieht deine Figur aus? Wie heißt sie? Wo kommt sie her? Was ist an
ihr interessant und besonders?

\subsubsection*{5. Motive}\label{motive-1}
\addcontentsline{toc}{subsubsection}{5. Motive}

Was will deine Figur? Was hindert sie daran? Was würde sie tun, um es zu
erreichen?

\subsubsection*{6. Beziehungen}\label{beziehungen-1}
\addcontentsline{toc}{subsubsection}{6. Beziehungen}

Woher kennen sich die Figuren? Welche Verbindungen gibt es zwischen
ihnen?

\section{Etwas tun}\label{etwas-tun}

Bei FU erzählst du gemeinsam mit deinen Mitspielerinnen spannende
Geschichten über eure Figuren. Es geht nicht ums Gewinnen oder um den
Wettbewerb miteinander, sondern darum, dass alle Freude daran haben, die
gemeinsame Geschichte zu entwickeln.

\subsection{Szenen und Runden}\label{szenen-und-runden}

Das Spiel unterteilt sich in Szenen und Runden. Eine Szene ist ein
Zeitabschnitt innerhalb der Geschichte, in dem es um eine bestimmte
Situation, einen Ort oder eine Gruppe von Figuren geht. Szenen sind die
Grundbausteine der Geschichte und können beliebig lange Zeiträume von
wenigen Sekunden und bis hin zu vielen Stunden abzubilden. Jede Szene
sollte eine bestimmte Zielsetzung verfolgen und enden, wenn diese
bearbeitet wurde. Szenen sollten die Handlung der Geschichte
vorantreiben, etwas über eine Figur verraten oder die beschriebenen
Ereignisse farbig ausschmücken. Viele Szenen tun das sogar alles
gleichzeitig.

Im Laufe der Szene beschreiben die Erzählerin und die Spielerinnen, was
die Figuren tun. Spielerinnen ``spielen'' ihre Figuren, sprechen für sie
und geben an, wie sie handeln. Die Erzählerin tut dasselbe für alle
anderen Figuren, Wesen und Monstren, die in der Szene vorkommen.

Der Ausgang von Szenen kann durch Würfelwürfe entschieden werden, aber
das ist nicht zwingend notwendig. Es ist durchaus möglich, dass das Ziel
einer Szene durch reines Rollenspiel und Interaktion zwischen den
Figuren erreicht wird.

Wenn es wichtig ist, was genau jede Figur tut und in welcher Reihenfolge
das passiert, wird die Szene in Runden unterteilt. Eine Runde ist ein
Zeitraum, der lang genug ist, dass jede Figur genau eine Handlung
durchführen kann, sei es ein Angriff, das Halten einer mitreißenden
Rede, das Werfen eines Gegenstandes zu einer Mitstreiterin, etwas auf
dem Smartphone nachzuschlagen oder irgendeine andere Aufgabe.

Die Spielerinnen sagen, was ihre Figuren tun, während die Erzählerin
entscheidet, wie die anderen Figuren und Wesen agieren. Nun entscheidet
ihr alle zusammen, in welcher Reihenfolge das alles passiert. Wenn jede
betroffene Figur die Gelegenheit zum Handeln hatte, endet die Runde.
Wenn nötig, beginnt danach eine neue.

Es sollte gesagt werden, dass nur die Spielerinnen würfeln. So hat die
Erzählerin Hände und Kopf frei, um neue Ränke und Komplotte zu schmieden
und sich auf das nächste aufregende Gefecht vorzubereiten.

\begin{center}\rule{0.5\linewidth}{\linethickness}\end{center}

\columnsbegin

\subsubsection{Wer plant Szenen?}\label{wer-plant-szenen}

Die Spielerinnen können Szenen vorschlagen, die sie gerne sehen würden
oder in denen sie mitspielen wollen, doch meistens entscheidet am Ende
die Erzählerin, welche Szenen tatsächlich gespielt werden und in welcher
Reihenfolge sie stattfinden.

Bevor die Erzählerin die abschließende Entscheidung trifft, könnt ihr
noch gemeinsam die Reihenfolge der Ereignisse ausdiskutieren, was
passiert, wo es passiert und wer dabei ist.

Manche Gruppen möchten auch jeder Spielerin reihum einmal die
Gelegenheit geben, eine Szene zu planen, und auch das ist völlig in
Ordnung.

\subsubsection{Wie plant man eine
Szene?}\label{wie-plant-man-eine-szene}

Die Szene zu planen bedeutet, du legst fest, wo und wann die Ereignisse
stattfinden, wer dabei ist, was kurz vorher passiert ist oder was gleich
passieren wird. Ort, Figuren, Ereignis - oder Wo, Wann, Wer und Was.

Beziehe beim Beschreiben der Szene alle Sinne mit ein und stelle
interessante oder wichtige Details zum Handlungsort und den anwesenden
Figuren heraus. Beachte beim Erstellen der Szene, welches Ziel sie
verfolgen soll.

\subsubsection{Was ist eine
Zielsetzung?}\label{was-ist-eine-zielsetzung}

Die Zielsetzung einer Szene kann alles mögliche sein, das eine Spielerin
oder Figur erreichen will. Figurenziele könnten unter anderem sein:
etwas in Erfahrung bringen, einen Feind bezwingen, mit jemandem reden,
eine kurze oder lange Reise unternehmen, sich auf einen Kampf
vorbereiten, jemanden hereinlegen oder etwas stehlen. Spielerinnen-Ziele
könnten sein: die eigene Figur in einem abgefahrenen Kampf sehen, das
Rätsel lösen, ein Geheimnis über die eigene Figur preisgeben oder mit
einer bestimmten Figur oder Spielerin interagieren. Oft überschneiden
sich die Zielsetzungen von Spielerinnen und Figuren auch.

\subsubsection{Müssen Szenen in der richtigen zeitlichen Reihenfolge
passieren?}\label{muxfcssen-szenen-in-der-richtigen-zeitlichen-reihenfolge-passieren}

Ihr könnt alle möglichen Erzähltechniken aus Literatur und Film
verwenden - Rückblenden, Vorausblenden, parallel laufende
Handlungsstränge und sogar Montage-Sequenzen (Zusammenschnitte kürzerer
aufeinanderfolgender Szenen zu einer langen). Obwohl die meisten eurer
Szenen wahrscheinlich zeitlich nacheinander stattfinden werden, braucht
ihr euch nicht davon einschränken lassen.

\subsubsection{Muss ich Runden
verwenden?}\label{muss-ich-runden-verwenden}

Runden sind optional, aber sie können nützlich sein, um die Zeit
einzuteilen, wenn mehrere Figuren gleichzeitig verschiedene Dinge tun
wollen. Ihr könnt sie bei Bedarf einsetzen, um Handlungen zu ordnen und
unterteilen. Bei einigen Szenen ergeben sich Runden fast von allein, bei
anderen denkt ihr vielleicht gar nicht daran.

\columnsend

\begin{center}\rule{0.5\linewidth}{\linethickness}\end{center}

\subsection{Ungerade Zahlen vermeiden}\label{ungerade-zahlen-vermeiden}

Wenn eine Figur eine Handlung versucht, bei der der Ausgang nicht
eindeutig oder völlig offensichtlich ist, macht ihre Spielerin einen
Würfelwurf, bei dem sie versucht, die ungeraden Zahlen zu vermeiden.

Um den Ausgang der Handlung zu entscheiden, werft ihr einen W6. Das Ziel
ist es, die ungeraden Zahlen zu vermeiden, also möglichst eine gerade
Zahl zu werfen. Je höher die gerade Zahl, desto besser das Ergebnis.
Wenn ihr eine ungerade Zahl werft, schlägt die Handlung entweder fehl,
oder sie war nicht ganz so gut wie nötig oder wie erhofft. Je niedriger
die ungerade Zahl, desto schlechter das Ergebnis. Weiter unten folgt
eine hilfreiche Tabelle, die das Konzept verdeutlicht.

Der Wurf zur Vermeidung der ungeraden Zahlen ist das Herzstück von FU.
Meistens sind gerade Zahlen gut und ungerade schlecht, aber die
tatsächlichen Ergebnisse hängen von der konkreten Situation ab. Es kann
zum Beispiel sein, dass der Wurf einer 1 nicht unbedingt ein unerhörter
Misserfolg ist, sondern eher der kleinstmögliche, schwächste Erfolg, den
man sich vorstellen kann.

\begin{quote}
In diesem Kapitel folgen wir den Irrungen und Wirrungen zweier Figuren
aus unterschiedlichen Spielwelten.

Sir Camden verfolgt den bösen Lord Kane zu Pferd. Er sieht, wie Lord
Kane über eine hohe Hecke springt und im dahinter liegenden Wald
verschwindet. Sir Camden versucht nun, ebenfall über die Hecke zu
springen, also werft ihr einen W6 und es fällt eine 2. Sir Camdens Pferd
überspringt die Hecke, aber Sir Camden selbst bekommt einen Stoß
versetzt und ist kurzzeitig durcheinander.

Captain Vance geht in Deckung, als ein erneuter Kugelhagel die Mauer
trifft, hinter der er sich versteckt. Er greift nach einem beschädigten
Funkgerät und betätigt ein paar Schalter, um zu versuchen, beim
Hauptquartier Unterstützung anzufordern. Ihr werft einen W6 und es fällt
eine 1. Vance findet die richtige Frequenz nicht, und ein Querschläger
trifft das Funkgerät. Es ist zerstört.
\end{quote}

\begin{longtable}[]{@{}ll@{}}
\toprule
\begin{minipage}[b]{0.07\columnwidth}\raggedright\strut
Wurf
\strut\end{minipage} &
\begin{minipage}[b]{0.87\columnwidth}\raggedright\strut
Bekommst du, was du willst?
\strut\end{minipage}\tabularnewline
\midrule
\endhead
\begin{minipage}[t]{0.07\columnwidth}\raggedright\strut
6
\strut\end{minipage} &
\begin{minipage}[t]{0.87\columnwidth}\raggedright\strut
\textbf{Ja, und \ldots{}} Du bekommst, was du willst, und noch etwas
mehr.
\strut\end{minipage}\tabularnewline
\begin{minipage}[t]{0.07\columnwidth}\raggedright\strut
4
\strut\end{minipage} &
\begin{minipage}[t]{0.87\columnwidth}\raggedright\strut
\textbf{Ja \ldots{}} Du bekommst, was du willst.
\strut\end{minipage}\tabularnewline
\begin{minipage}[t]{0.07\columnwidth}\raggedright\strut
2
\strut\end{minipage} &
\begin{minipage}[t]{0.87\columnwidth}\raggedright\strut
\textbf{Ja, aber \ldots{}} Du bekommst, was du willst, aber es hat einen
Preis.
\strut\end{minipage}\tabularnewline
\begin{minipage}[t]{0.07\columnwidth}\raggedright\strut
5
\strut\end{minipage} &
\begin{minipage}[t]{0.87\columnwidth}\raggedright\strut
\textbf{Nein, aber \ldots{}} Du bekommst nicht, was du willst, aber es
ist kein völliger Verlust.
\strut\end{minipage}\tabularnewline
\begin{minipage}[t]{0.07\columnwidth}\raggedright\strut
3
\strut\end{minipage} &
\begin{minipage}[t]{0.87\columnwidth}\raggedright\strut
\textbf{Nein \ldots{}} Du bekommst nicht, was du willst.
\strut\end{minipage}\tabularnewline
\begin{minipage}[t]{0.07\columnwidth}\raggedright\strut
1
\strut\end{minipage} &
\begin{minipage}[t]{0.87\columnwidth}\raggedright\strut
\textbf{Nein, und \ldots{}} Du bekommst nicht, was du willst, und es
wird noch schlimmer.
\strut\end{minipage}\tabularnewline
\bottomrule
\end{longtable}

\textbar{}

\begin{center}\rule{0.5\linewidth}{\linethickness}\end{center}

\subsubsection{Geschlossene Fragen}\label{geschlossene-fragen}

Bei FU verwendet ihr zum Entscheiden, wie eine Handlung ausgeht, Fragen
in geschlossener Form. Eine geschlossene Frage kann man nur mit ``Ja''
oder ``Nein'' beantworten. Wenn du in eine Situation gerätst, die mit
Würfeln geklärt werden muss, schlage eine geschlossene Frage vor:
``Springe ich über den Abgrund?'', ``Geb ich dem Trottel eins auf die
Nase?'', ``Fällt die Schankmaid auf meinen ungezwungenen Charme und mein
gewinnendes Lächeln herein?'' Der Würfelwurf wird deine Frage
beantworten und deine Reaktion leiten.

Oft muss die Frage gar nicht ausdrücklich gestellt werden, sondern
ergibt sich aus der Handlung, die du gerade versuchst: ``Du nimmst
Anlauf und springst am Rande des Abgrunds ab. Würfeln!''

\subsubsection{Fragen-Alternativen}\label{fragen-alternativen}

Ihr könnt auch andere Fragen stellen, wenn ihr mögt, müsst dann aber die
Ergebnistabelle ändern. Eine naheliegende Frage wäre ``Wie gut gelingt
es mir?'' Das könnte zu den folgenden Ereignissen führen:

\begin{longtable}[]{@{}ll@{}}
\toprule
Wurf & Wie gut gelingt es mir?\tabularnewline
\midrule
\endhead
6 & Legendärer Erfolg\tabularnewline
4 & Völliger Erfolg\tabularnewline
2 & Gerade so geschafft\tabularnewline
5 & Gerade so daneben\tabularnewline
3 & Völliger Misserfolg\tabularnewline
1 & Sagenhafter Misserfolg, und zwar richtig!\tabularnewline
\bottomrule
\end{longtable}

Ihr könnt euch gerne eigene Fragen und Antworten ausdenken, die zu eurer
Gruppe, eurem Spiel und eurer Geschichte passen.

\subsubsection{Würfelergebnis-Alternativen}\label{wuxfcrfelergebnis-alternativen}

Manchen Spielerinnen gefallen die geraden/ungeraden Ergebnisse nicht -
viele bevorzugen 1-3 als schlechte und 4-6 als gute Ergebnisse. In
diesem Falle würde die Ergebnistabelle so aussehen:

\begin{longtable}[]{@{}ll@{}}
\toprule
Wurf & Bekommst du, was du willst?\tabularnewline
\midrule
\endhead
6 & Ja, und\ldots{}\tabularnewline
5 & Ja\ldots{}\tabularnewline
4 & Ja, aber\ldots{}\tabularnewline
3 & Nein, aber\ldots{}\tabularnewline
2 & Nein\ldots{}\tabularnewline
1 & Nein, und\ldots{}\tabularnewline
\bottomrule
\end{longtable}

\begin{center}\rule{0.5\linewidth}{\linethickness}\end{center}

\subsection{Erfolg \& Misserfolg}\label{erfolg-misserfolg}

Wenn du die Würfel wirfst, hat deine Figur entweder Erfolg bei dem, was
sie versucht, oder es schlägt fehl. Normalerweise genügt das, um die
Geschichte voranzutreiben, doch es können auch andere Dinge passieren.

Wenn du eine Handlung versuchst, stellst du die Frage: ``Erreicht meine
Figur was sie will?'' Darauf gibt es sechs mögliche Antworten:

\begin{quote}
Ja, und\ldots{}

Ja\ldots{}

Ja, aber\ldots{}

Nein, aber\ldots{}

Nein\ldots{}

Nein, und\ldots{}
\end{quote}

\emph{Ja} und \emph{Nein} sind klar - sie sagen euch, ob die Handlung
erfolgreich war oder nicht. Das \emph{und} und das \emph{aber} sind
einschränkende Bezeichner dafür, wie groß Erfolg oder Misserfolg sind.
Wenn du eine solche Einschränkung würfelst, machst du eine zusätzliche
Aussage darüber, wie die Handlung erfolgreich ist oder fehlschlägt.
Diese Zusatzaussage kann einer Figur einen Zustand auferlegen oder ein
Detail zu einer Szene hinzufügen.

\textbf{Zustände}: Das sind körperliche, geistige oder gesellschaftliche
Auswirkungen, die Einfluss darauf haben, wie eine Figur sich verhält
oder wie sie Handlungen angeht. Zustände umfassen Dinge wie
\emph{wütend}, \emph{verwirrt}, \emph{müde} und \emph{bewusstlos}.
Einige Zustände stehen schon auf den FU-Figurenbögen und es ist auch
genug Platz, damit ihr selbst eigene hinzufügen könnt.

\textbf{Details}: Dies sind Merkmale der Umgebung oder Szene, die sich
durch den Ausgang einer Handlung verändern können. Details können z.B.
Feuer fangende Vorhänge, zersplitternde Fenster, davonlaufende Tiere
oder angehaltene Maschinen sein. Details sind immer eng verknüpft mit
der Szene und der versuchten Handlung.

\begin{center}\rule{0.5\linewidth}{\linethickness}\end{center}

\subsubsection{Beispiele für Erfolg \&
Misserfolg}\label{beispiele-fuxfcr-erfolg-misserfolg}

Kehren wir zu unserem Beispiel von vorhin zurück und schauen, was für
jede der möglichen Antworten auf die Frage \textbf{``Überspringt Sir
Camden die Hecke?''} passieren würde:

\textbf{Ja, und} er holt Lord Kane ein. Dies ist ein Detail, das die
Szene verändert.\\
\textbf{Ja,} er springt über die Hecke. Hier gibt es keine
und/aber-Einschränkung, also werden weder Zustand noch Detail
hinzugefügt.\\
\textbf{Ja, aber} Sir Camden verliert die Orientierung und ist
vorübergehend \textbf{verwirrt}. Dies ist ein Zustand, der der Figur
auferlegt wird.\\
\textbf{Nein, aber} er entdeckt weiter hinten eine Lücke in der Hecke.
Dieses Detail gibt der Figur eine andere Möglichkeit, die Verfolgung
fortzuführen.\\
\textbf{Nein,} das Pferd scheut vor dem Absprung. Hier gibt es keine
und/aber-Einschränkung, also werden weder Zustand noch Detail
hinzugefügt.\\
\textbf{Nein, und} sein Pferd bäumt sich auf und wirft ihn ab. Beim
Sturz auf den Boden \textbf{verletzt} er sich. Das ist ein Zustand.

\subsubsection{Wer entscheidet über Zustände \&
Details?}\label{wer-entscheidet-uxfcber-zustuxe4nde-details}

Alle können Zustände oder Details vorschlagen, die ihrer Meinung nach
zur versuchten Handlung oder zum Würfelergebnis passen. Normalerweise
überlegt sich die Spielerin, die gewürfelt hat, zusammen mit der
Erzählerin eine angemessen dramatische Auswirkung. Schöner ist es
jedoch, wenn alle am Tisch ihre coolen Ideen einbringen.\\
Wenn es darum geht, welcher Zustand oder welches Detail am Ende auf das
Ergebnis angewendet wird, hat die Erzählerin aber immer das letzte Wort.

\subsubsection{Wann soll ich Zustände
verwenden?}\label{wann-soll-ich-zustuxe4nde-verwenden}

Wie bei allen Einschränkungen hängt das von der Situation ab. In den
Beispielen weiter oben werden der aktiven Spielerin Zustände auferlegt,
wenn es nicht ganz glatt für sie läuft (\textbf{Ja, aber/Nein, und}).
Die Zustände machen der Figur nach einem geringen Erfolg oder sogar
völligen Misserfolg das Leben etwas schwerer. Du kannst auch einer
anderen Figur Zustände auferlegen, wenn sie das Ziel einer Handung war,
die gut für deine Figur ausging. Wenn du mit einem Beamten diskutierst
und ein \textbf{Ja, und}-Ergebnis bekommst, kannst du den Zustand
\textbf{verwirrt} auf die Zielperson anwenden. Wenn du versuchst, einem
Feind davonzurennen und ein \textbf{Nein, aber}-Ergebnis bekommst,
könnte er dich einholen, aber dafür den Zustand \textbf{müde} erhalten.
Wenn ihr Zustände so benutzt, haben die Figuren der Spielerinnen im
weiteren Verlauf der Szene wahrscheinlich einen Vorteil.

\subsubsection{Wann soll ich Details
benutzen?}\label{wann-soll-ich-details-benutzen}

Füge Details hinzu, wenn die versuchte Handlung die Szene oder Umgebung
irgendwie ändert. Das könnten veränderte Machtverhältnisse in der Szene
sein (``\textbf{Ja,} er springt über die Hecke \textbf{und} holt Lord
Kane ein.''), oder eine geänderte Umgebung (``\textbf{Nein, aber} er
entdeckt eine Lücke in der Hecke.'').\\
Details werden oft angewendet, wenn die aktive Figur einen gewissen
Vorteil erlangt (\textbf{Ja, und/Nein, aber}). Besonders effektvoll sind
sie, wenn ihr damit Situationen unterhaltsamer und/oder gefährlicher
macht: ``Schwingst du dich am Kronleuchter durch den Raum? \textbf{Ja,
aber} es lösen sich Kerzen und setzen die Taverne in Brand.''

Details können, abhängig von den Umständen, sofortige oder dauerhafte
Wirkung haben. Eine Lücke in der Hecke kann sofort benutzt werden, um
die Verfolgung fortzusetzen; eine brennende Taverne is eine dauerhafte
Gefahr, bis jemand das Feuer löscht!

\begin{center}\rule{0.5\linewidth}{\linethickness}\end{center}

\subsection{Modifikatoren}\label{modifikatoren}

Manchmal machen äußere Umstände, Ausrüstung oder Fähigkeiten eine
Handlung leichter oder schwerer. Modifikatoren verändern die Anzahl der
Würfel, die du werfen darfst, wenn du etwas tun willst.

\textbf{Dinge einfacher nachen}: Wirf je einen zusätzlichen Würfel für
alle Beschreiber, Ausrüstungsgegenstände, Zustände und Details, die
einen Vorteil für die gemachte Handlung bringen. Das Ergebnis ist der
beste gefallene Würfel (nicht unbedingt der mit dem höchsten Wert), nach
Wahl der Spielerin.

\begin{quote}
Im Wald versucht Sir Camden, Lord Kane aufzuspüren. Du merkst an, dass
der Ritter ein \textbf{guter Jäger} ist, daher bekommst du einen
weiteren W6. Du wirfst 2W6 und es fallen eine 5 und eine 4. Du nimmst
die 4 und es wird entschieden, dass Sir Camden nach kurzer Suche die
Spuren von Lord Kane findet und ihnen zu einer finsteren Zitadelle folgt
\ldots{}
\end{quote}

\textbf{Dinge schwieriger machen}: Wirf je einen zusätzlichen Würfel für
alle Beschreiber, Ausrüstungsgegenstände, Zustände und Details, die
deine Handlung erschweren. Das Ergebnis ist der schlechteste gefallene
Würfel.

\begin{quote}
Captain Vance entschliesst sich, zu einem in der Nähe stehendem
Geländewagen zu rennen. Die Erzählerin weist darauf hin, dass es
zwischen Vances Versteck und dem Fahrzeug \textbf{keine Deckung} gibt.
``Gelangt Vance unverletzt zun Geländewagen?'' Du wirfst 2W6 und es
fallen eine 3 und eine 5. Du musst die 3 nehmen und Captain Vance wird
beim Erreichen des Geländewagens \textbf{verletzt} \ldots{}
\end{quote}

\textbf{Aufhebung}: Je ein helfender Würfel hebt einen behindernden
Würfel auf, so dass du nie ``negative'' und ``positive'' Würfel
gleichzeitig wirfst.

\begin{quote}
Im weiteren Verlauf seiner Suche muss Sir Camden eine \textbf{steile}
(-) Klippe erklimmen. Der Ritter ist \textbf{stark} (+) und hat ein Seil
(+). Insgesamt ergibt das einen einzelnen Bonuswürfel für diese Handlung
(die steile Klippe und die Stärke des Ritters heben sich auf und es
bleibt nur das Seil übrig). Du wirfst 2W6 und es fallen eine 3 und eine
6.
\end{quote}

\begin{center}\rule{0.5\linewidth}{\linethickness}\end{center}

\subsubsection{Beispiel-Modifikatoren}\label{beispiel-modifikatoren}

Captain Vance rast im Geländewagen dahin, als eine Wache versucht, ihn
durch das Fenster hinauszuziehen. ``Kann Vance die Wache abschütteln?''
Es ist \textbf{schwer} (+), Vance durchs Fenster zu ziehen, aber er ist
\textbf{verwundet} (-) und \textbf{überrascht} (-), und der Angreifer
ist \textbf{sehr stark} (-). Insgesamt musst du 2 Strafwürfel werfen. Du
wirfst 3W6 und es fallen 2, 4 und 3. Du musst das schlechteste Ergebnis
nehmen, also die 3. Der Wache gelingt es, Vance aus dem Geländewagen zu
bugsieren.

\subsubsection{Andere Würfe?}\label{andere-wuxfcrfe}

Bei FU gibt es keine ``Gegenwürfe'' oder ``Wettstreit-Aktionen'',
``Schadenswürfe'' oder ``Trefferwürfe''. Der Wurf zum Vermeiden der
ungeraden Zahlen ist die einzige Wurfart, die du bei FU machst, egal ob
du versuchst, mit dem Auto durch das Gedränge eines Einkaufszentrums zu
fahren, mit einem Riesen zu ringen oder die Verletzung aus einem
Querschläger zu ignorieren.

\subsubsection{\texorpdfstring{Wie funktionieren
``Gegenaktionen''?}{Wie funktionieren Gegenaktionen?}}\label{wie-funktionieren-gegenaktionen}

Zunächst einmal gilt, dass immer nur die Spielerinnen würfeln. Zuerst
zählst du alle Vorteile (+) und Nachteile (-) zusammen, die für deine
Figur gelten. Dann prüfst du alle Faktoren, die für den Gegner gelten
und fügst sie ebenfalls zu deinem Wurf hinzu. Ist der Gegner schwach?
Dann bekommst du einen Bonuswürfel. Ist der Gegner Weltmeister im
Armdrücken? Dann rechne einen Strafwürfel ein. Und so weiter, bis alles
eingerechnet ist. Wenn du weißt, wie viele Bonus- oder Strafwürfel du
hast, würfle. Ist das Ergebnis eine gerade Zahl, gewinnst du den
Wettbewerb; ist es eine ungerade Zahl, ist dein Gegner im Vorteil.

\subsubsection{An deine Handlungen näher
herangehen}\label{an-deine-handlungen-nuxe4her-herangehen}

Schwertkämpfe, politische Debatten, Weltraumrennen, internationale
Kriege, Streitgespräche oder Feuergefechte werden alle mit dem
Gerade-Ungerade-Wurf abgehandelt. Das Entscheidende ist, dass ihr die
Handlungen nach Bedarf aus der Nähe oder aus der Ferne betrachtet, und
zwar durch die Art der Frage, die durch das Würfeln beantwortet werden
soll. Bei einen dramatischen Schwertkampf kannst du einen Schlagabtausch
mit mehrmaligem hin und her erreichen, wenn du fragst ``Treffe ich den
Grafen De Montief?''. Du könntest aber auch den gesamten Kampf mit einem
einzelnen Wurf abhandeln, indem du fragst ``Besiege ich den Grafen De
Montief im Duell?''. Oder du verlegst den Kampf auf eine höhere Ebene
und fragst ``Stürmen meine Landsknechte die Burg des Grafen De
Montief?'' Stelle den Fokus ganz nach Bedarf flexibel ein!

\subsubsection{Option: Paschwürfe}\label{option-paschwuxfcrfe}

Der Wurf von zweimal, dreimal oder sogar viermal der gleichen Zahl kann
den Erfolg viel größer oder Misserfolg viel schlimmer machen. Wenn der
Wert deines Erfolgswürfels mehrmals gefallen ist, sind die Auswirkungen
viel dramatischer. Du darfst pro gefallenem gleichem Würfel eine ``und
\ldots{}''-Aussage hinzufügen. Wenn die Handlung fehlschlägt, muss diese
Aussage die Situation verschlimmern. Ist die Handlung erfolgreich,
verbessert jede der zusätzlichen Aussagen die Situation. Zum Beispiel
steht Sir Camden den Schergen von Lord Kane gegenüber. ``Besiegt Lord
Kane die Schergen?'' Er ist verletzt und in der Unterzahl, also wirfst
du zwei Strafwürfel. Du würfelst und es fallen 3, 3 und 2. Du musst das
schlechteste Ergebnis nehmen - die doppelte 3! Normalerweise wäre das
ein \textbf{``Nein \ldots{}''}-Ergebnis, doch die doppelte 3 macht es zu
einem \textbf{``Nein, und \ldots{}''} Hättest du dreimal die 3 geworfen,
wäre das Ergebnis ein \textbf{``Nein, und \ldots{} und \ldots{}''}
gewesen!

\begin{center}\rule{0.5\linewidth}{\linethickness}\end{center}

\subsection{FU-Punkte}\label{fu-punkte}

FU-Punkte sind Hilfsmittel, die du wie eine Währung ausgeben kannst, um
die Erfolgschancen deiner Figur zu erhöhen. Es gibt zwei Möglichkeiten,
sie einzusetzen:

\textbf{Bonuswürfel:} Gib vor dem Würfelwurf einen FU-Punkt aus, um
einen einzelnen Bonuswürfel hinzuzubekommen. Dieser wirkt wie jeder
andere Bonuswürfel. Du kannst so viele Bonuswürfel hinzunehmen, wie du
FU-Punkte hast, musst sie aber alle zur gleichen Zeit ansagen.

\textbf{Neuwurf:} Gib nach dem Würfelwurf einen FU-Punkt aus, um einen
einzelnen Würfel erneut zu werfen. Der zweite Wurf zählt dann aber -
neugewürfelte Würfel dürfen nicht ein weiteres Mal neu geworfen werden.
Du kannst so viele Würfel neu werfen, wie du FU-Punkte hast, musst sie
aber alle zur gleichen Zeit ansagen - bevor der erste Würfel neu
geworfen wird.

Du kannst bei einer einzelnen Handlung auch mehrere FU-Punkte für diese
zwei Möglichkeiten ausgeben, entweder für eine allein oder für beide. Es
ist also ausdrücklich erlaubt, vor dem Wurf einen FU-Punkt auszugeben,
um einen Bonuswürfel hinzuzufügen, und nach dem Wurf einen zu bezahlen,
um einen Würfel neu zu werfen.

\begin{quote}
Wieder im Hauptquartier angekommen, versucht Captain Vance, General
Wallace dazu zu bringen, die Raketentruppen unter seinen Befehl zu
stellen und auszusenden. Vance hat in dieser Situation, was Beschreiber
angeht, nicht viel zu bieten, also gibst du 2 FU-Punkte für den Wurf
aus. Du wirfst den Grundwürfel und die zwei Bonuswürfel und es fallen 1,
1 und 3. Damit bist du nicht zufrieden und gibst deinen letzten FU-Punkt
für einen Neuwurf aus. Du nimmst eine der Einsen und wirfst den Würfel
erneut \ldots{}
\end{quote}

\textbf{FU-Punkte verdienen:} FU-Punkte verdienen sich die Spielerinnen,
indem sie coole Dinge machen und ihre Rolle gut spielen. Immer wenn du
etwas tust, bei dem das Spiel stillsteht und alle ``Oooh!'' rufen, laut
über deine Eskapaden lachen, oder irgendetwas anderes, das die anderen
für belohnenswert halten, bringt dir das einen FU-Punkt ein.

\begin{longtable}[]{@{}l@{}}
\toprule
\begin{minipage}[t]{0.05\columnwidth}\raggedright\strut
\#\#\# FU-Punkte beim Sitzungsbeginn
\strut\end{minipage}\tabularnewline
\begin{minipage}[t]{0.05\columnwidth}\raggedright\strut
Bevor ihr losspielt, solltet ihr ausdiskutieren, wie viele FU-Punkte ihr
jeweils zu Beginn des Spiels bekommt. Je mehr FU-Punkte die Spielerinnen
am Anfang des Spiels haben, desto leichter wird es ihnen fallen, Erfolge
zu würfeln. Wenn ihr ein Spiel voller riskanter Abenteuer und
waghalsiger Actionszenen wollt, sind 2 FU-Punkte für den Spielbeginn
angemessen. Übermächtige Superheldinnen könnten sogar 3 FU-Punkte
bekommen. In schmutzigeren, realistischen Spielen könnte jede Spielerin
mit nur einem oder sogar gar ganz ohne FU-Punkt beginnen.
\strut\end{minipage}\tabularnewline
\begin{minipage}[t]{0.05\columnwidth}\raggedright\strut
\textless{}!-- Starting FU Points? e number of FU Points you begin a
game with should be discussed before play begins. e more FU Points play-
ers begin with, the more easily they will achieve successes. If playing
games of high adventure or over-the-top action, it would be reasonable
to begin with 2 FU Points. Super-powered heroes might begin with as many
as 3 FU Points. For grittier games each player might only begin with 1
FU Point, or even none.
\strut\end{minipage}\tabularnewline
\begin{minipage}[t]{0.05\columnwidth}\raggedright\strut
Combien de points de FU au départ ?
\strut\end{minipage}\tabularnewline
\begin{minipage}[t]{0.05\columnwidth}\raggedright\strut
Le nombre de points de FU avec lesquels votre personnage commence doit
être discuté avant le début de la partie. Plus vous aurez de points,
plus vous aurez de chance d'avoir des succès. Si vous êtes impliqué dans
des aventures rocambolesque, il est raisonnable de commencer avec 2
points de FU. Des super-héros peuvent débuter avec 3 points. Dans un jeu
plus corsé, les joueurs peuvent démarrer avec un seul point de
FU\ldots{} voire aucun ! --\textgreater{}
\strut\end{minipage}\tabularnewline
\begin{minipage}[t]{0.05\columnwidth}\raggedright\strut
\#\#\# Darf ich FU-Punkte verschenken oder teilen? Das könnt ihr selbst
entscheiden. Die Standardregel ist ``nein'', aber \ldots{}
\strut\end{minipage}\tabularnewline
\begin{minipage}[t]{0.05\columnwidth}\raggedright\strut
\textless{}!-- Can I give / share FU Points? at is up to you and your
group. e default is ``no'', but\ldots{}
\strut\end{minipage}\tabularnewline
\begin{minipage}[t]{0.05\columnwidth}\raggedright\strut
Est-ce que je peux donner / partager mes points ?
\strut\end{minipage}\tabularnewline
\begin{minipage}[t]{0.05\columnwidth}\raggedright\strut
C'est à vous de voir avec votre groupe. La règle par défaut dit ``non'',
mais\ldots{} --\textgreater{}
\strut\end{minipage}\tabularnewline
\begin{minipage}[t]{0.05\columnwidth}\raggedright\strut
\#\#\# Bekommt die Erzählerin auch FU-Punkte? Üblicherweise nicht, aber
sie kann einem besonders starken Gegner oder Monster einen, zwei oder
drei zugestehen.
\strut\end{minipage}\tabularnewline
\begin{minipage}[t]{0.05\columnwidth}\raggedright\strut
\textless{}!-- Do Narrators get FU points? Usually, no, but they might
allow a powerful villain or monster to have one, two or three.
\strut\end{minipage}\tabularnewline
\begin{minipage}[t]{0.05\columnwidth}\raggedright\strut
Les Narrateurs ont-ils droit à des points de FU ?
\strut\end{minipage}\tabularnewline
\begin{minipage}[t]{0.05\columnwidth}\raggedright\strut
Normalement, non, mais on peut allouer entre un et trois points à un
méchant particulièrement puissant, ou à un monstre. --\textgreater{}
\strut\end{minipage}\tabularnewline
\begin{minipage}[t]{0.05\columnwidth}\raggedright\strut
\#\#\# Andere Möglichkeiten, FU-Punkte zu verdienen
\strut\end{minipage}\tabularnewline
\begin{minipage}[t]{0.05\columnwidth}\raggedright\strut
Manchmal möchtet ihr vielleicht ändern, wie man FU-Punkte verdient.
Eventuell verdient man sie, indem man Monstren den Todesstoß versetzt,
das Ziel seiner Figur erreicht, oder mehrere Einsen oder Sechsen
würfelt.
\strut\end{minipage}\tabularnewline
\begin{minipage}[t]{0.05\columnwidth}\raggedright\strut
So lassen sich effektiv Stimmung oder Stil des Spiels variieren. Wollt
ihr ein düsteres Höhlenlabyrinth voller Monstren und Schätze erkunden?
Dann bekommen die Spielerinnen FU für das Töten von Monstren. In einem
schmutzigen Endzeitszenario erhalten sie FU, wenn sie im Kampf gegen
``die da oben'' Boden verlieren. Entscheidet euch, ob das die einzige
Möglichkeit sein soll, FU zu verdienen, oder ob es zusätzlich zur
normalen Belohnung für gutes Rollenspielen sein soll. Diese Entscheidung
wirkt sich auch dramaturgisch auf die Spielstimmung aus. Probiert aus,
was ihr damit anfangen könnt.
\strut\end{minipage}\tabularnewline
\begin{minipage}[t]{0.05\columnwidth}\raggedright\strut
\textless{}!-- Other ways to earn FU You might like to change the way
you earn FU Points. You might earn them for dealing the killing blow to
monsters, achieving your character's goal, or roll- ing multiple 1's or
6's. is is a really good way to change the tone or style of the game.
Want to play a dungeon crawl? Reward FU for killing monsters. A gritty
dystopian game? Reward FU when the characters lose out to e Man. Decide
if these are the only ways to earn FU, or if they are in addition to the
normal roleplaying rewards. is decision will also have a dramatic e ect
on the tone of play. See what you can do with it.
\strut\end{minipage}\tabularnewline
\begin{minipage}[t]{0.05\columnwidth}\raggedright\strut
Autres moyens pour gagner des FU
\strut\end{minipage}\tabularnewline
\begin{minipage}[t]{0.05\columnwidth}\raggedright\strut
Vous pouvez choisir de changer le moyen de gagner des points de FU. Vous
pourriez les gagner en tuant des monstres, en achevant un des buts de
votre personnage, ou en lançant un multiple 1 ou un multiple 6.
\strut\end{minipage}\tabularnewline
\begin{minipage}[t]{0.05\columnwidth}\raggedright\strut
C'est un bon outil pour changer le ton ou le style du jeu. Si vous
voulez faire de l'exploration de donjon, récompensez les joueurs quand
ils tuent un monstre. Vous êtes plongé dans une cruelle dystopie ?
Attribuez des points de FU quand les personnages perdent leur bataille
contre leur Ennemi. Décidez si ce sont les seuls moyens de gagner du FU
ou s'ils s'ajoutent aux récompenses habituelles. Cette décision aura
aussi un effet dramatique sur le ton de la partie. Choisissez ce que
vous voulez en faire. --\textgreater{}
\strut\end{minipage}\tabularnewline
\begin{minipage}[t]{0.05\columnwidth}\raggedright\strut
\#\#\# Andere Möglichkeiten, FU einzusetzen
\strut\end{minipage}\tabularnewline
\begin{minipage}[t]{0.05\columnwidth}\raggedright\strut
Experimentiert ein wenig mit FU-Punkten und passt ihren Einsatz an eure
Spielgruppe oder eure Geschichte an. Hier ein paar Vorschläge:
\strut\end{minipage}\tabularnewline
\begin{minipage}[t]{0.05\columnwidth}\raggedright\strut
\textless{}!-- Other ways to use FU Play around with FU Points. Experi-
ment, or adjust the ways you use them to suit your gaming group or the
type of story you are playing. Here are some suggestions;
\strut\end{minipage}\tabularnewline
\begin{minipage}[t]{0.05\columnwidth}\raggedright\strut
Autres moyens de dépenser du FU
\strut\end{minipage}\tabularnewline
\begin{minipage}[t]{0.05\columnwidth}\raggedright\strut
Amusez-vous avec les points de FU. Expérimentez, modifiez-en l'usage en
fonction de votre groupe de joueurs ou du type d'histoire que vous
voulez interpréter. Voici quelques suggestions : --\textgreater{}
\strut\end{minipage}\tabularnewline
\begin{minipage}[t]{0.05\columnwidth}\raggedright\strut
\#\#\#\# FU als Lebenspunkte
\strut\end{minipage}\tabularnewline
\begin{minipage}[t]{0.05\columnwidth}\raggedright\strut
Statt einen veränderlichen Vorrat von FU-Punkten zu haben, beginnt jede
Figur mit 3 Punkten. Sie können ganz normal ausgegeben, aber auch
verloren werden, wenn die Figur Schaden an Körper oder Geist nimmt
(Verletzungen, Erschöpfung, Furcht, etc). In diesem Fall kannst du FU
wieder auffüllen (zurück auf 3), indem du gemeinsam mit einer anderen
Figur eine Szene (ohne Würfeln) spielst, die etwas über eure Beziehung
zeigt.
\strut\end{minipage}\tabularnewline
\begin{minipage}[t]{0.05\columnwidth}\raggedright\strut
\textless{}!-- FU as Health Rather than having a varying and changeable
number of FU points, every character begins with 3. ey can be spent as
normal, but can also be lost when the character su ers physical or
mental stress (injuries, fatigue, fear, etc). FU in this instance can be
refreshed (brought back up to 3) by roleplaying out a (non-dice rolling)
scene with an- other character that reveals something about your
relationship.
\strut\end{minipage}\tabularnewline
\begin{minipage}[t]{0.05\columnwidth}\raggedright\strut
FU comme des Points de Vie
\strut\end{minipage}\tabularnewline
\begin{minipage}[t]{0.05\columnwidth}\raggedright\strut
Au lieu d'avoir un nombre de points de FU variable, tous les personnages
commencent avec 3 points. Ils peuvent être dépensés comme les autres,
mais également perdus quand les personnages subissent des dommages
physiques ou mentaux (blessures, fatigue, peur, etc\ldots{}). Dans ces
cas, les points de FU peuvent être récupérés (pour revenir au maximum de
3) en interprétant une scène (sans jet de dés) avec un autre personnage
qui lèvera le voile sur leur relation. --\textgreater{}
\strut\end{minipage}\tabularnewline
\begin{minipage}[t]{0.05\columnwidth}\raggedright\strut
\#\#\#\# Alles neu würfeln
\strut\end{minipage}\tabularnewline
\begin{minipage}[t]{0.05\columnwidth}\raggedright\strut
Gib einen FU-Punkt aus, um alle deine Würfel erneut zu werfen. Es gilt
dabei ``alles oder nichts''; du kannst also nicht ein paar gute
Würfelergebnisse behalten und den Rest neu würfeln. Verwendet diese
Variante statt der normalen Neuwurfregel.
\strut\end{minipage}\tabularnewline
\begin{minipage}[t]{0.05\columnwidth}\raggedright\strut
\textless{}!-- Re-Roll Everything Spend a FU point to re-roll all your
dice. is is an all-or-nothing thing, so you can't keep a couple of good
results and roll the rest. Use this variant in- stead of the normal
re-roll rule.
\strut\end{minipage}\tabularnewline
\begin{minipage}[t]{0.05\columnwidth}\raggedright\strut
Tout relancer
\strut\end{minipage}\tabularnewline
\begin{minipage}[t]{0.05\columnwidth}\raggedright\strut
En dépensant un point de FU, vous pouvez relancer tous vos dés. C'est
tout ou rien, aussi vous ne pourrez pas garder vos bons jets et relancer
les autres. Utilisez cette variante au lieu de la relance classique.
--\textgreater{}
\strut\end{minipage}\tabularnewline
\begin{minipage}[t]{0.05\columnwidth}\raggedright\strut
\#\#\#\# Ein Auge verdrehen
\strut\end{minipage}\tabularnewline
\begin{minipage}[t]{0.05\columnwidth}\raggedright\strut
Gib einen FU-Punkt aus, um einen einzelnen Würfel um einen Wert (ein
Würfelauge) herauf- oder herunterzudrehen. Diese Option ist
verlässlicher als ein Neuwurf, da du damit immer ein ``Nein'' zu einem
``Ja'' machen kannst. Wenn ihr diese Variante verwendet, solltet ihr die
normale Neuwurfregel nicht benutzen.
\strut\end{minipage}\tabularnewline
\begin{minipage}[t]{0.05\columnwidth}\raggedright\strut
\textless{}!-- Flip a Pip Spend a FU Point to adjust a single die up or
down by one pip. Spend mul- tiple FU points to adjust a die multiple
pips. is option is more reliable than a re-roll as you will always be
able to turn a ``no'' into a ``yes''. If you used this variant the
normal re-roll rule should not be used.
\strut\end{minipage}\tabularnewline
\begin{minipage}[t]{0.05\columnwidth}\raggedright\strut
Changer un point
\strut\end{minipage}\tabularnewline
\begin{minipage}[t]{0.05\columnwidth}\raggedright\strut
On peut dépenser un point de FU pour augmenter ou diminuer le résultat
obtenu d'un point. Cette option est plus fiable qu'une relance parce
qu'elle peut changer un ``non'' en ``oui''. Si vous utilisez cette
variante, la relance classique peut être ignorée. --\textgreater{}
\strut\end{minipage}\tabularnewline
\begin{minipage}[t]{0.05\columnwidth}\raggedright\strut
\#\#\#\# Eine Requisite benutzen
\strut\end{minipage}\tabularnewline
\begin{minipage}[t]{0.05\columnwidth}\raggedright\strut
Gib einen FU-Punkt aus, um einen normalen Gegenstand für die Dauer der
Szene als Ausrüstungsgegenstand benutzen zu können.
\strut\end{minipage}\tabularnewline
\begin{minipage}[t]{0.05\columnwidth}\raggedright\strut
\textless{}!-- Use a Prop Spend a FU point to turn a prop into an item
of Gear for the duration of the scene.
\strut\end{minipage}\tabularnewline
\begin{minipage}[t]{0.05\columnwidth}\raggedright\strut
Utiliser un équipement
\strut\end{minipage}\tabularnewline
\begin{minipage}[t]{0.05\columnwidth}\raggedright\strut
En dépensant un point de FU, vous pourrez transformer un élément
d'équipement simple en Matériel pour la durée de la scène.
--\textgreater{}
\strut\end{minipage}\tabularnewline
\begin{minipage}[t]{0.05\columnwidth}\raggedright\strut
\#\#\#\# Talente und Stärken
\strut\end{minipage}\tabularnewline
\begin{minipage}[t]{0.05\columnwidth}\raggedright\strut
Gib Figuren Sonderfähigkeiten, besondere Fertigkeiten und Stärken, die
sie nur benutzen können, wenn sie einen FU-Punkt dafür ausgeben. Sie
sollten mächtiger als Beschreiber sein - mit ihnen sollte man die
normalen Spielregeln brechen können oder eine übermenschliche Gabe
erhalten, wie zum Beispiel Fliegen, Gedankenlesen oder Teleportieren.
\strut\end{minipage}\tabularnewline
\begin{minipage}[t]{0.05\columnwidth}\raggedright\strut
\textless{}!-- Stunts and Powers Give characters special abilities,
skills or powers that can only be used by spending a FU Point. ese
should be more powerful than Descriptors - make them break the rules or
give a superhu- man knack, like the ability to y or read minds,
teleport, or whatever.
\strut\end{minipage}\tabularnewline
\begin{minipage}[t]{0.05\columnwidth}\raggedright\strut
Prouesses et Pouvoirs
\strut\end{minipage}\tabularnewline
\begin{minipage}[t]{0.05\columnwidth}\raggedright\strut
Donnez des capacités spéciales à vos personnages, ou des pouvoirs qui ne
peuvent être utilisées qu'en dépensant un point de FU. Ils doivent être
plus puissants que des Descripteurs - faites-en des outils surpuissants,
capables de passer outre une règle, comme la capacité de voler, de lire
les esprits ou de se téléporter, par exemple. --\textgreater{}
\strut\end{minipage}\tabularnewline
\begin{minipage}[t]{0.05\columnwidth}\raggedright\strut
\#\#\#\# Einen Treffer einstecken
\strut\end{minipage}\tabularnewline
\begin{minipage}[t]{0.05\columnwidth}\raggedright\strut
Anstatt zu würfeln, sage, dass du ``einen Treffer einsteckst''. Alle
Würfel, die du geworfen hättest, werden als Einsen behandelt (ja, wenn
du fünf Würfel geworfen hättest, zählen sie als fünf Einsen!). Als Lohn
für dein Leiden erhältst du einen FU-Punkt.
\strut\end{minipage}\tabularnewline
\begin{minipage}[t]{0.05\columnwidth}\raggedright\strut
\textless{}!-- Taking a Hit Instead of rolling, declare you are ``taking
a hit''. All the dice you should have rolled are treated as if they
rolled 1's (yes, if you are holding 5 dice, they count as ve 1's!). In
return for your suf- fering you earn a FU Point.
\strut\end{minipage}\tabularnewline
\begin{minipage}[t]{0.05\columnwidth}\raggedright\strut
Prendre un coup
\strut\end{minipage}\tabularnewline
\begin{minipage}[t]{0.05\columnwidth}\raggedright\strut
Au lieu de lancer les dés, annoncez que vous ``prenez un coup''. Tous
les dés sont considérés comme des ``1'' (oui, si vous avez 5 dés, ça
fera cinq fois le score 1 !). En compensation de votre peine, vous
gagnez un point de FU. --\textgreater{}
\strut\end{minipage}\tabularnewline
\bottomrule
\end{longtable}

\newpage

\subsection{Handlungen kurzgefasst}\label{handlungen-kurzgefasst}

\subsubsection{1. Die Szene planen
\{unnumbered\}}\label{die-szene-planen-unnumbered}

Wo findet die Szene statt? Wer ist vor Ort? Was wollen sie erreichen?
Muss die Szene in Runden unterteilt werden?

\subsubsection{2. Handeln \{unnumbered\}}\label{handeln-unnumbered}

Spiele deine Figur und beschreibe was passiert. Arbeite auf das Ziel
hin.

\subsubsection{3. Eine Frage stellen
\{unnumbered\}}\label{eine-frage-stellen-unnumbered}

Müsst ihr den Ausgang eines Konflikts klären oder willst du eine
Handlung machen, stelle eine geschlossene Frage (z.B. ``Habe ich
Erfolg?'')

\subsubsection{4. Modifikatoren einrechnen
\{unnumbered\}}\label{modifikatoren-einrechnen-unnumbered}

+1 Bonuswürfel für jeden Beschreiber, jeden Ausrüstungsgegenstand, jedes
Detail der Umgebung, jeden Zustand oder andere Merkmale, die die
Handlung erleichtern.

+1 Strafwürfel für jeden Beschreiber, jeden Ausrüstungsgegenstand, jedes
Detail der Umgebung, jeden Zustand oder andere Merkmale, die die
Handlung erschweren.

\subsubsection{5. Würfeln \{unnumbered\}}\label{wuxfcrfeln-unnumbered}

Wirf alle deine Würfel. Das Ergebnis ist der beste (wenn du Bonuswürfel
wirfst) oder der schlechteste (wenn du Strafwürfel wirfst) gefallene
Würfel.

\subsubsection{6. Ergebnis beschreiben
\{unnumbered\}}\label{ergebnis-beschreiben-unnumbered}

Benutzt das Würfelergebnis zum Beschreiben, wie der Konflikt oder die
Handlung ausgeht. Wendet bei Bedarf Zustände oder Umgebungsdetails an.

\begin{longtable}[]{@{}l@{}}
\toprule
\begin{minipage}[t]{0.05\columnwidth}\raggedright\strut
\strut\end{minipage}\tabularnewline
\begin{minipage}[t]{0.05\columnwidth}\raggedright\strut
\strut\end{minipage}\tabularnewline
\begin{minipage}[t]{0.05\columnwidth}\raggedright\strut
\# Die Erzählerin
\strut\end{minipage}\tabularnewline
\begin{minipage}[t]{0.05\columnwidth}\raggedright\strut
\strut\end{minipage}\tabularnewline
\begin{minipage}[t]{0.05\columnwidth}\raggedright\strut
Hier findest du einfache Ratschläge für Erzählerinnen. Sie sind ziemlich
allgemein gehalten, weil die konkreten Details sich auf das gewählte
Genre, die Stimmung und den Spielstil beziehen, für die ihr euch als
Gruppe entscheiden werdet.
\strut\end{minipage}\tabularnewline
\begin{minipage}[t]{0.05\columnwidth}\raggedright\strut
\strut\end{minipage}\tabularnewline
\begin{minipage}[t]{0.05\columnwidth}\raggedright\strut
\strut\end{minipage}\tabularnewline
\begin{minipage}[t]{0.05\columnwidth}\raggedright\strut
\#\# Reden ist gut
\strut\end{minipage}\tabularnewline
\begin{minipage}[t]{0.05\columnwidth}\raggedright\strut
\strut\end{minipage}\tabularnewline
\begin{minipage}[t]{0.05\columnwidth}\raggedright\strut
Beginnt euer Spiel mit einer Unterhaltung. Sprecht darüber, was du und
deine Mitspielerinnen erwarten und an welcher Stelle sich eure
Vorstellungen decken. In diesen Bereichen der Überschneidung werdet ihr
wahrscheinlich später im Spiel den größten Spaß haben. Ihr solltet
\emph{mindestens} das Genre und die angestrebte Stimmung eurer
gemeinsamen Geschichte zusammen besprechen. Es liegen zum Beispiel
Welten zwischen Gothic-Horror- und Splatterfilmen.
\strut\end{minipage}\tabularnewline
\begin{minipage}[t]{0.05\columnwidth}\raggedright\strut
Sprecht auch während des Spiels weiter miteinander. Ermuntere deine
Mitspielerinnen, Ideen auszutauschen und ihre Meinung zu Szenen, Zielen
und Herausforderungen beizusteuern. Falls du Klärungsbedarf zu Ideen,
Themen oder Problemen hast, dann frage nach.
\strut\end{minipage}\tabularnewline
\begin{minipage}[t]{0.05\columnwidth}\raggedright\strut
\textless{}!--(Parler, c'est bien) Démarrez votre jeu par le dialogue.
Parlez de ce que les joueurs aimeraient retirer du jeu, ce que vous
voulez, et où ces idées se rejoignent. Ces points communs ont de fortes
chances d'être les moments où tout le monde s'amusera. \emph{A minima},
vous devrez oeuvrer ensemble pour déterminer le genre et le ton de votre
histoire. Dialoguez sur tout, pour que tout le monde soit en accord avec
les spécificités (il y a un monde entre les styles ``horreur gothique''
et ``gore'', par exemple).
\strut\end{minipage}\tabularnewline
\begin{minipage}[t]{0.05\columnwidth}\raggedright\strut
Parlez aussi pendant le jeu. Encouragez les joueurs à partager leurs
pensées et fournir leurs idées dans les scènes, pour établir leurs buts
et découvrir les défis auxquels leurs personnages feront face. Si vous
avez besoin de clarifier vos idées, le thème ou lever des problèmes,
faites-le. --\textgreater{}
\strut\end{minipage}\tabularnewline
\begin{minipage}[t]{0.05\columnwidth}\raggedright\strut
\#\# Zuhören ist gut
\strut\end{minipage}\tabularnewline
\begin{minipage}[t]{0.05\columnwidth}\raggedright\strut
\strut\end{minipage}\tabularnewline
\begin{minipage}[t]{0.05\columnwidth}\raggedright\strut
Wo miteinander gesprochen wird, sollte auch zugehört werden. Hör zu,
wenn deine Spielerinnen dir etwas sagen, sei direkt es Gespräch oder
durch die Taten ihrer Figuren. Arbeite Dinge ein, die mit den
Beschreibern auf ihren Figurenbögen zu tun haben, denn das sind genau
die Dinge, die die Spielerinnen cool finden.
\strut\end{minipage}\tabularnewline
\begin{minipage}[t]{0.05\columnwidth}\raggedright\strut
\strut\end{minipage}\tabularnewline
\begin{minipage}[t]{0.05\columnwidth}\raggedright\strut
\strut\end{minipage}\tabularnewline
\begin{minipage}[t]{0.05\columnwidth}\raggedright\strut
\#\# Im Spiel selbst
\strut\end{minipage}\tabularnewline
\begin{minipage}[t]{0.05\columnwidth}\raggedright\strut
\strut\end{minipage}\tabularnewline
\begin{minipage}[t]{0.05\columnwidth}\raggedright\strut
Plane vor dem Leiten eines Spiels nicht zu weit voraus. Die
Würfelergebnisse werden einen Gutteil der Handlung mitbestimmen und
auftretende Lücken füllt ihr durch Zustände und Umgebungsdetails. Hier
sind einige Tipps, die dir helfen das Spiel am Laufen zu halten und
genau so viel Spaß zu haben wie die Spielerinnen.
\strut\end{minipage}\tabularnewline
\begin{minipage}[t]{0.05\columnwidth}\raggedright\strut
\strut\end{minipage}\tabularnewline
\begin{minipage}[t]{0.05\columnwidth}\raggedright\strut
\textbf{Weniger ist mehr:} Verkompliziere deine Aufgabe nicht, indem du
allzu viele Handlungsstränge oder verschachtelte Wendungen einführst.
Das Durcheinander ergibt sich ganz von selbst und die Geschichte wird in
Richtungen abschweifen, auf die du alleine nie gekommen wärst.
\strut\end{minipage}\tabularnewline
\begin{minipage}[t]{0.05\columnwidth}\raggedright\strut
\strut\end{minipage}\tabularnewline
\begin{minipage}[t]{0.05\columnwidth}\raggedright\strut
\textbf{Du bist nicht der Feind:} Das erscheint dir vielleicht
offensichtlich, soll hier aber ausdrücklich erwähnt werden. Deine
Aufgabe besteht darin, die Geschichte in interessante Richtungen zu
lenken - nicht darin, alle umzubringen. Du kannst deinen Spielerinnen
Hinweise geben und Vorschläge machen, wenn du glaubst, dass es die
Geschichte verbessert. Manchmal stellst du die Gegenspieler dar und
solltest dich dabei nicht zurückhalten, aber trotzdem fair bleiben.
Manchmal wiederum stellst du eine Verbündete, Begleiterin oder
Mitstreiterin dar und auch dabei solltest du fair und im Sinne des
Spiels handeln.
\strut\end{minipage}\tabularnewline
\begin{minipage}[t]{0.05\columnwidth}\raggedright\strut
\strut\end{minipage}\tabularnewline
\begin{minipage}[t]{0.05\columnwidth}\raggedright\strut
**Sag „ja``:** Wenn Spielerinnen Vorschläge machen oder Fragen stellen,
tun sie das wahrscheinlich, weil sie interessant finden, was gerade
passiert. Sie haben wahrscheinlich eine coole Idee für die Geschichte.
Ermuntere sie und sag getrost''ja" zu solchen Eingaben. Das soll nicht
heißen, dass die Spielerinnen alles bekommen sollten, was sie möchten --
wenn es passt, solltest du sie aber Elemente hinzufügen lassen.
\strut\end{minipage}\tabularnewline
\begin{minipage}[t]{0.05\columnwidth}\raggedright\strut
\strut\end{minipage}\tabularnewline
\begin{minipage}[t]{0.05\columnwidth}\raggedright\strut
\textbf{Gib Würfelwürfen Bedeutung:} Immer, wenn du um einen Würfelwurf
bittest, sollte danach etwas Interessantes passieren. Und zwar
unabhängig vom Würfelergebnis! Zwinge deine Spielerinnen nicht zum
Würfeln, wenn das Ergebis für den Fortgang der Geschichte unbedeutend
ist oder wenn eine Niederlage den Schwung der Handlung stoppen würde.
\strut\end{minipage}\tabularnewline
\begin{minipage}[t]{0.05\columnwidth}\raggedright\strut
\textless{}!-- (En jeu) Pendant le jeu, inutile de planifier trop de
choses à l'avance. Les lancers de dés guideront l'action et vous et vos
joueurs aurez à remplir les blancs en y introduisant des Conditions ou
des Détails. Voici quelques astuces pour garder l'élan et continuer à
vous amuser autant que les joueurs.
\strut\end{minipage}\tabularnewline
\begin{minipage}[t]{0.05\columnwidth}\raggedright\strut
\textbf{Restez simple} : Ne vous compliquez pas le travail en ajoutant
des sous-intrigues et trop de retournements de situation alambiqués. Les
choses se compliquent aisément toutes seules et l'histoire atteindra des
sommets au-delà de votre imagination par elle-même.
\strut\end{minipage}\tabularnewline
\begin{minipage}[t]{0.05\columnwidth}\raggedright\strut
\textbf{Vous n'êtes pas l'ennemi} : Cela va sans dire, mais ça va mieux
en le disant. Votre rôle est de guider l'histoire pour qu'elle aille
dans des directions intéressantes ; il n'est pas de tuer tout le monde.
Vous pouvez donner à vos joueurs des indices ou des idées si vous pensez
que ça peut générer une histoire plus plaisante. Parfois vous décrirez
les adversaires, avec fougue, mais équité. D'autres fois, vous serez un
allié, un compagnon, un égal, mais une fois encore, il faudra l'être en
accord avec l'esprit du jeu.
\strut\end{minipage}\tabularnewline
\begin{minipage}[t]{0.05\columnwidth}\raggedright\strut
\textbf{Dites ``oui''} : Si les joueurs font des suggestions ou posent
des questions, c'est probablement qu'ils sont intéressés par ce qui se
passe. Ils ont peut-être une bonne idée à intégrer à l'histoire.
Encouragez-les et dites ``oui'' en toute confiance à leurs demandes.
Cela ne signifie pas qu'il faille laisser les joueurs faire ce qu'ils
veulent - mais que vous devez laisser les joueurs introduire des
éléments à l'histoire quand c'est le bon moment.
\strut\end{minipage}\tabularnewline
\begin{minipage}[t]{0.05\columnwidth}\raggedright\strut
\textbf{Les lancers doivent avoir un sens} : Chaque fois que vous
réclamez un lancer de dés il doit se passer quelque chose d'intéressant,
QUEL QUE SOIT LE RÉSULTAT. Ne faites pas faire de jets de dés si le
résultat n'a pas d'importance pour l'histoire, ou si l'échec de l'action
met un coup d'arrêt au déroulement de l'histoire.--\textgreater{}
\strut\end{minipage}\tabularnewline
\bottomrule
\end{longtable}

\subsection{Drei Fragen zur
Orientierung}\label{drei-fragen-zur-orientierung}

Wenn ihr ein Spiel beginnt, sprich mit den Spielerinnen über ihre
Erwartungen. Versuche die folgenden drei Fragen zu beantworten:

\subsubsection{Was sollen die Figuren
tun?}\label{was-sollen-die-figuren-tun}

Wollen die Spielerinnen Türen eintreten, Monster umbringen und
bergeweise Schätze nach Hause schleppen? Wollen sie sich wie große
Heldinnen fühlen? Oder wie Außenseiterinnen, die gegen unbezwingbar
scheinende Widerstände ankämpfen?

\subsubsection{Wie wollen die Spielerinnen sich
fühlen?}\label{wie-wollen-die-spielerinnen-sich-fuxfchlen}

Wollen sie das Gefühl haben, die Welt zu verändern? Dass ihre Figuren an
Geld, Ruhm oder Macht immer reicher werden? Oder wollen sie das Gefühl
haben, mit dem Rücken zur Wand zu stehen und ständig in Lebensgefahr zu
schweben?

\subsubsection{Was ist die Rolle der
Erzählerin?}\label{was-ist-die-rolle-der-erzuxe4hlerin}

Welche Herausforderungen, Begegnungen und Situationen wirst du den
Figuren präsentieren, um das oben Angesprochene zu erreichen? Sorgst du
dafür, dass jede Herausforderung eine großzügigen Belohnung mitbringt?
Führst du jeden Kampf hart, aber fair? Verfolgt der Gegenspieler seine
Ziele unermüdlich und kompromisslos?

Beantwortet diese Fragen und alle am Tisch werden sich über ihre Rolle
im kommenden Abenteuer im Klaren sein.

\begin{center}\rule{0.5\linewidth}{\linethickness}\end{center}

\subsection{Erholung und Heilung}\label{erholung-und-heilung}

Im Spielverlauf können Figuren Verletzungen erleiden, körperlichen oder
geistigen Belastungen ausgesetzt sein oder von einer ganzen Reihe
anderer Zustände heimgesucht werden. Wann und wie schnell sich die
Figuren von Zuständen erholen sollte immer zur Geschichte passen.
Normalerweise geschieht das einfach durch das Verstreichen der Zeit, das
muss aber nicht so sein. Eine gute Faustregel ist, dass Figuren sich
zwischen zwei Szenen von einem oder mehreren Zuständen erholen können.
Das hängt aber natürlich auch davon ab, wie viel Zeit zwischen den
Szenen vergeht.

\subsection{Belohnungen}\label{belohnungen}

Belohne Spielerinnen mit FU-Punkten für gutes Rollenspiel und das
Erreichen von Zielen. Du kannst sie auch für eine Reihe anderer Gründe
belohnen, die die Gruppe untereinander abspricht (siehe dazu die
\textbf{Anmerkungen/Seitenleiste} zu FU-Punkten in Kapitel 4).

Mit FU-Punkten können die Erfolgschancen der Figuren verbessert werden;
daher stellen sie eine schöne, direkte und greifbare Belohnung dar. Du
brauchst jedoch nicht die einzige sein, die Belohnungen verteilt. Alle
sollten sich melden, wenn sie finden, dass jemand etwas cooles, witziges
oder beeindruckendes getan hat, und einen FU-Punkt dafür ausloben.

\textbf{Wie viel und wie oft?} Sei mit Belohnungen großzügig. Die
positive Verstärkung durch das Verdienen von FU-Punkten wird die
Spielerinnen, je nachdem, zu weiteren heldenhaften, spektakulären oder
verderbten Taten anregen. Berücksichtige auch, ob euer Spiel nur über
eine Sitzung gehen soll oder Teil einer längeren Geschichte oder
Kampagne wird. In Spielen mit nur einer Sitzung brauchen die
Spielerinnen ihre FU-Punkte wahrscheinlich schneller auf, daher sollten
sie auch schneller wieder aufgefüllt werden.

\subsection{Figuren und Hindernisse}\label{figuren-und-hindernisse}

Alle Nichtspielerfiguren, Monster, Fallen, Bösewichte,
Geländeeigenschaften, Wesen und Hindernisse, denen die Spielerinnen
begegnen, werden im Großen und Ganzen genau wie Figuren definiert. Es
gibt keine Regeln, die dich beim Gestalten von Figuren oder Hindernisse
einschränken. Die einzige Voraussetzung ist, dass sie unterhaltsam und
interessant sind.

\begin{center}\rule{0.5\linewidth}{\linethickness}\end{center}

\subsubsection{Die Schale}\label{die-schale}

Eine Methode zum Ausgeben von FU-Punkten ist es, eine Schale mit
Glassteinen, Pappplättchen oder Spielmarkern in die Tischmitte zu
stellen. Jeder Gegenstand darin entspricht einem FU-Punkt. Die
Erzählerin kann Spielerinnen bitten, sich ``einen aus der Schale'' zu
nehmen, und andere Spielerinnen können sich daraus bedienen, wenn sie
jemanden belohnen möchten. Diese Möglichkeit setzt Vertrauen zwischen
Spielerinnen und Erzählerin voraus, erlaubt aber auch einen
reibungslosen Spielablauf, da niemand erst nachfragen muss ``Gibt es
dafür einen FU-Punkt?''

\subsubsection{Erfahrung gewinnen}\label{erfahrung-gewinnen}

In FU geht es nicht um das Aufsteigen auf immer höhere ``Level''. Die
Figuren können zwar allerlei Erfahrungen machen und aus ihnen auch
lernen, aber der eigentliche Fortschritt entsteht durch die sich
verändernde Welt und Geschichte.

Gegebenenfalls können die Spielerinnen zwischen den Spielsitzungen einen
ihrer Beschreiber ändern. Diese Änderung sollte mit einer gerade in der
Geschichte gemachten Erfahrung zusammenhängen. Ob Ausrüstung zwischen
den Sitzungen gewechselt werden kann, liegt im Ermessen der Erzählerin.

Auch Motive können sich von Sitzung zu Sitzung verändern. Gib deinen
Spielerinnen zu Beginn jeder Spielsitzung Zeit, über ihre Ziele
nachzudenken und sie wenn nötig anzupassen oder neue zu schreiben.

\subsubsection{Den Überblick über Hindernisse
behalten}\label{den-uxfcberblick-uxfcber-hindernisse-behalten}

Benutze Haftnotizen oder Karteikarten, um Einzelheiten über deine
Figuren, Monster und Hindernisse festzuhalten. Notiere ihre Beschreiber,
Ausrüstung und andere Daten. Wenn sie Zustände auferlegt bekommen,
vermerke auch diese auf ihrer Karte.

Halte auch wichtige Geländemerkmale auf Karteikarten fest. Notiere alle
zugehörigen Beschreiber, so dass die Spielerinnen sie in ihre Planungen
einbeziehen können.

\subsubsection{Welche Informationen teilst du mit deinen
Spielerinnen?}\label{welche-informationen-teilst-du-mit-deinen-spielerinnen}

\emph{I'm HERE}

Manche Gruppen und Erzählerinnen spielen ganz offen und machen um die
Beschreiber und Details von Kreaturen und Monstern keinen Hehl. Andere
wiederum halten solche Informationen zurück. Beide Herangehensweisen
erzeugen eine unterschiedliche Spielweise.

Mit „offenen Karten" zu spielen bedeutet, dass alle Bescheid wissen, was
passiert und welche Möglichkeiten für tolle Szenen und Aktionen es gibt.
Die Spielerinnen sehen, welche Ausrüstung, Zustände und Beschreiber
gerade im Spiel sind und können sie in Szenen einbeziehen.

Die Einzelheiten von Bösewichten, Zielen und anderen Hindernissen
geheimzuhalten wird die Spielerinnen dazu zwingen, ihre Gegner
auszuloten, die Umgebung zu erforschen und verschiedene
Herangehensweisen auszuprobieren. Sie können geschickt taktieren, um
sich selbst gut zu positionieren oder die Situation in Richtungen
steuern, die sie für vorteilhaft halten. Es ist immer sehr befriedigend,
wenn man mit Bonuswürfeln belohnt wird, weil man den Beschreiber eines
Gegners richtig eingeschätzt hat.

\section{Race to the Temple of Tot}\label{race-to-the-temple-of-tot}

\section{Das Rennen zum Tempel des
Toth}\label{das-rennen-zum-tempel-des-toth}

\begin{quote}
Der berühmte Entdecker Tennessee Smith ist auf einer heißen Spur, die
ihn zur altertümlichen Götzenstatue von Toth führen soll, einem
Gegestand, dem übernatürliche Kräfte nachgesagt werden.
Unglücklicherweise ist Giles Fishburne, Smiths Erzfeind, der Reliquie
ebenfalls auf der Spur und hat sich mit den Nazis zusammengetan!
\end{quote}

Dieses kurze Abenteuer bringt euch schnell mitten ins Geschehen. Es
zeigt beispielhaft, wie du selbst eigene Abenteuer, Gegenspieler und
Hindernisse vorbereiten kannst. Am Ende des Abenteuers findet ihr
vorgefertigte Figuren, die ihr benutzen könnt. \emph{Das Rennen zum
Tempel des Toth} besteht aus einer Reihe von Schlüsselbegegnungen, die
du nach Belieben verwenden, neu anordnen oder ignorieren kannst. Denke
daran, dass die Würfelwürfe zu allen möglichen, interessanten Wendungen
führen werden, und sobald die Figuren unterwegs sind, ist alles möglich!

\subsection{Vor dem Spiel}\label{vor-dem-spiel}

Sprecht vor dem Spiel kurz über Stimmung und Stil, in denen ihr spielen
wollt. Das Szenario gehört mit seiner haarsträubenden Action, den
ruchlosen Schurken und überzeichneten Helden ins Pulpgenre. Vergewissere
dich, dass das allen klar ist: redet darüber, was an diesem Genre cool
ist und welche Aspekte von Genrefilmen wie \emph{Indiana Jones},
\emph{Die Mumie} oder \emph{Rocketeer} euch gefallen haben.

Lies den obigen Stimmungstext vor und lasse alle Mitspielerinnen ein
paar Ideen für coole Sachen einbringen, die sie während des Abenteuers
erleben möchten. Schreibe dir alle Ideen auf; falls das Spiel mal zäh
werden sollte oder du nicht weiter weißt, baue eine davon ein!

\subsection{Szenen}\label{szenen}

Bei den hier vorgestellten Szenen findest du jeweils eine Beschreibung
des Handlungsortes und Vorschläge für Beschreiber, die du oder deine
Spielerinnen einbauen könnt. Der kursive Teil legt den Grundstein für
die Handlung und gibt das Ziel vor. Umschreibe deinen Spielerinnen die
Details in eigenen Worten.

\subsubsection{Herausforderungen}\label{herausforderungen}

Die Herausforderungen und Gegenspieler dieses Abenteuers sind in Kästen
mit passenden Beschreibern, Zuständen und Hinweisen zusammengefasst. Sie
sollen dir eine Hilfe sein, aber du musst sie nicht zwingend benutzen.
Deine eigenen coolen Ideen sollten immer Vorrang haben vor dem, was hier
geschrieben steht!

\subsubsection{Flucht vom Bulak-Markt}\label{flucht-vom-bulak-markt}

\textbf{Szene:} Zentralasien -- ein geschäftiger Markt in der Stadt
Bulak zur Mittagszeit. Den innere Marktbereich umgeben hohen Stein- und
Lehmbauten mit schmalen Bogenfenstern. Enge, verwinkelte Gassen voller
Stände, Kneipen, Kaffee- und Warenhäuser führen in alle
Himmelsrichtungen.

\textbf{Beschreiber:} Überfüllte Straßen, exotische Waren

\begin{quote}
Die Figuren haben gerade eine Karte zum Tempel von Toth erstanden.
Dummerweise sind nun Nazihandlanger aufgetaucht, die ebenfalls die Karte
haben wollen. Es sind eine Menge Schläger, und obwohl die Figuren gegen
sie kämpfen könnten, wäre es wohl einfacher zu fliehen. Können sie
entkommen?
\end{quote}

\begin{longtable}[]{@{}rl@{}}
\toprule
Gegenspieler & Nazischlägertypen\tabularnewline
\midrule
\endhead
\textbf{Beschreiber} & Zahlreich, Muskulöse Arier, Nicht sehr
schlau\tabularnewline
\textbf{Ausrüstung} & Laute Maschinenpistolen\tabularnewline
\textbf{Zustände} & □ Verwirrt, □ Eingekreist, □ Aufgehalten, □ Außer
Gefecht\tabularnewline
\bottomrule
\end{longtable}

\textbf{Hinweise:} Es gibt so viele Schlägertrupps wie
Spielerinnenfiguren.

 \textbf{Was kann schiefgehen?} Die Figuren werden gefangen; die Karte
geht verloren.

\subsubsection{Die Übersetzung der
Karte}\label{die-uxfcbersetzung-der-karte}

\textbf{Szene:} Ein dunkler Cafésalon, ein Kaffeehaus oder ein
Hinterzimmer irgendwo in der Stadt Bulak. Der Duft starken Kaffees und
exotischer Lebensmittel durchströmt den Raum.

\textbf{Beschreiber:} Abgeschieden.

\begin{quote}
Tennessee Smith und seine Begleiter sind im Besitz der Karte, müssen sie
nun aber übersetzen und die seltsamen Symbole und Beschriftungen
entziffern. Können sie die Karte entschlüsseln?
\end{quote}

\begin{longtable}[]{@{}rl@{}}
\toprule
Herausforderung & Die Übersetzung der Karte\tabularnewline
\midrule
\endhead
\textbf{Beschreiber} & Uralte Schriftzeichen, brüchiges
Papier\tabularnewline
\textbf{Zustände} & □ Zerrissen, □ Zerfallen, □ Verschmutzt, □ Zu Asche
verbrannt\tabularnewline
\bottomrule
\end{longtable}

\textbf{Was kann schiefgehen?} Die Karte wird beschädigt; die Figuren
missverstehen die Karte; sie können die Karte nicht entziffern und
brauchen die Hilfe eines Experten für alte Sprachen.

\subsubsection{Die Gou-Zou-Schlucht}\label{die-gou-zou-schlucht}

\textbf{Szene:} Die majestätische Gou-Zou-Schlucht, 800 Meter tief und
von einer einsamen Eisenträger-Bahnbrücke überspannt. Zu beiden Seiten
der Schlucht liegt weites, offenes Flachland und in der Ferne kan man
das Kau-Gebirge sehen.

\textbf{Beschreiber:} Weite Ebenen, Unpassierbare Schlucht.

\begin{quote}
Die Figuren folgen den Hinweisen auf der Karte bis zur Gou-Zou-Schlucht,
entweder per Automobil oder auf dem Pferd. Giles Fishburne ist ihnen an
Bord des deutschen Luftschiffs ``Der Vogel'' hart auf den Fersen. Werden
die Figuren die Brücke unbeschadet überqueren und ihren Vorsprung
gegenüber den Deutschen halten?
\end{quote}

\begin{longtable}[]{@{}rl@{}}
\toprule
Gegenspieler & Nazi-Flugtruppen\tabularnewline
\midrule
\endhead
\textbf{Beschreiber} & Sie fliegen!, Beweglich\tabularnewline
\textbf{Ausrüstung} & Störanfälliges Fluggeschirr, Tödliche
Stielhandgranate\tabularnewline
\textbf{Zustände} & □ Verwirrt, □ Gestrandet, □ Aufgehalten, □ Außer
Gefecht\tabularnewline
\bottomrule
\end{longtable}

\textbf{Hinweise:} Es gibt so viele Flugtruppen wie Figuren.

\begin{longtable}[]{@{}rl@{}}
\toprule
Herausforderung & Gou-Zou-Brücke\tabularnewline
\midrule
\endhead
\textbf{Beschreiber} & Ohne Zug drauf breit genug, mit Zug ziemlich
schmal\tabularnewline
\textbf{Zustände} & □ Wacklig, □ In die Luft gejagt\tabularnewline
\bottomrule
\end{longtable}

\textbf{Hinweise:} Dir ist klar, dass du einen heranrasenden Zug in die
Geschichte einbauen musst, oder?

\textbf{Was kann schiefgehen?} Die Brücke wird zerstört; die Karte geht
verloren; die Figuren werden gefangengenommen.

\subsubsection{Der Tempel}\label{der-tempel}

\textbf{Szene:} Ein uralter, in einen Berghang eingelassener Tempel.
Gewaltige Statuen unheilvoll wirkender Götter und Dämonen stehen rundum
an den Wänden aufgereiht. Alles ist von einer Staubschicht überzogen.
Tunnel führen tiefer in den Berg hinein.

\textbf{Beschreiber:} Dunkel, Still

\begin{quote}
Die Figuren müssen ins Innere des Tempels vordringen, allerdings warnt
sie die Karte vor hinterhältigen Fallen und schrecklichen Wächtern.
Werden sie die Hindernisse überwinden?
\end{quote}

\begin{longtable}[]{@{}rl@{}}
\toprule
Herausforderung & Tükische Fallen\tabularnewline
\midrule
\endhead
\textbf{Beschreiber} & Verborgen, Tödlich\tabularnewline
\textbf{Ausrüstung} & Giftpfeile, Rostige Speere\tabularnewline
\textbf{Zustände} & □ Entschärft, □ Ausgelöst, □ Entdeckt\tabularnewline
\bottomrule
\end{longtable}

\textbf{Hinweise:} Stelle den Figuren ein oder zwei Fallen. Erzähle
ihnen, dass der Weg mit Fallen versehen ist, aber sage ihnen nicht was
sie auslöst oder wo sind sind, bis sie zuschnappen!

\begin{longtable}[]{@{}rl@{}}
\toprule
\begin{minipage}[b]{0.19\columnwidth}\raggedleft\strut
Gegenspieler
\strut\end{minipage} &
\begin{minipage}[b]{0.75\columnwidth}\raggedright\strut
Steinwächter
\strut\end{minipage}\tabularnewline
\midrule
\endhead
\begin{minipage}[t]{0.19\columnwidth}\raggedleft\strut
\textbf{Beschreiber}
\strut\end{minipage} &
\begin{minipage}[t]{0.75\columnwidth}\raggedright\strut
Groß, Schwer, Langsam, Unermüdlich
\strut\end{minipage}\tabularnewline
\begin{minipage}[t]{0.19\columnwidth}\raggedleft\strut
\textbf{Ausrüstung}
\strut\end{minipage} &
\begin{minipage}[t]{0.75\columnwidth}\raggedright\strut
Schwere Steinwaffen
\strut\end{minipage}\tabularnewline
\begin{minipage}[t]{0.19\columnwidth}\raggedleft\strut
\textbf{Zustände}
\strut\end{minipage} &
\begin{minipage}[t]{0.75\columnwidth}\raggedright\strut
□ Beschädigt, □ Aus dem Gleichgewicht, □ Verlangsamt, □ Außer Gefecht
\strut\end{minipage}\tabularnewline
\bottomrule
\end{longtable}

\textbf{Hinweise:} Für je zwei Spielerfiguren gibt es einen
Steinwächter. Seine großen, schweren Waffen können mehrere Figuren
gleichzeitg treffen.

\textbf{Was kann schiefgehen?} Die Figuren können die Fallen nicht
überwinden; die Figuren werden durch die Fallen verletzt.

\subsubsection{Die Kammer des Götzen}\label{die-kammer-des-guxf6tzen}

\textbf{Szene:} Eine große Steinkammer, die durch eine geniale Anordnung
von Sonnenlicht reflektierenden Spiegeln beleuchtet wird. Der Götze von
Toth, die goldene Statue eines siebenköpfigen Affen, ruht auf einem
Steinsockel in der Mitte der Kammer.

\textbf{Beschreiber:} Große Bronzespiegel.

\begin{quote}
Die Figuren betreten die Kammer des Götzen und müssen feststellen, dass
Giles Fishburne und seine Nazihandlanger schon vor Ort sind! Werden sie
Giles besiegen und mit dem Götzen entkommen?
\end{quote}

\begin{longtable}[]{@{}rl@{}}
\toprule
Gegenspieler & Giles Fishburne\tabularnewline
\midrule
\endhead
\textbf{Beschreiber} & Gerissen, Arrogant, Schwertkämpfer\tabularnewline
\textbf{Ausrüstung} & Polierter Säbel\tabularnewline
\textbf{Zustände} & □ Verwirrt, □ Verletzt, □ Aufgehalten, □ Außer
Gefecht\tabularnewline
\bottomrule
\end{longtable}

\textbf{Hinweise:} Giles ist zwar ein gieriger, arroganter Schurke, aber
nicht dumm. Wenn es schlecht für ihn aussieht, wird er versuchen zu
fliehen, ein Bündnis auszuhandeln und/oder jeden hintergehen, der ihm im
Weg steht. Er ist den Nazis gegenüber nicht loyal!

\subsection{Figuren}\label{figuren}

Four pre-made characters can be found on the next page. They are each
examples of classic pulp charac- ter archetypes. players may tweak or
adjust them at the Narrator's discretion, before play starts, perhaps
changing a Descriptor or two, or switching out an item of Gear. Before
starting the game each player should define one or two relationships,
describing how the characters know each other. A quick and easy way to
do this is for everyone to describe how they know the character to their
left.

Auf der nächsten Seite findet ihr vier vorgefertigte Figuren, alle
Beispiele für klassische Pulp-Klischees. Die Spielerinnen können sie vor
dem Spiel in Absprache mit der Erzählerin abändern, zum Beispiel durch
Abwandlung eines Beschreibers oder indem sie einen Ausrüstungsgegenstand
austauschen. Außerdem sollte jede Spielerin ein bis zwei Beziehungen
festlegen, die beschreiben, woher die Figuren sich kennen. Am
einfachsten geht das, wenn alle beschreiben, woher sie die Person links
von sich kennen.

\subsubsection{Tennessee Smith, Tollkühner
Entdecker}\label{tennessee-smith-tollkuxfchner-entdecker}

\begin{longtable}[]{@{}rl@{}}
\toprule
\begin{minipage}[b]{0.14\columnwidth}\raggedleft\strut
Figur
\strut\end{minipage} &
\begin{minipage}[b]{0.80\columnwidth}\raggedright\strut
Tennessee Smith
\strut\end{minipage}\tabularnewline
\midrule
\endhead
\begin{minipage}[t]{0.14\columnwidth}\raggedleft\strut
\textbf{Beschreiber}
\strut\end{minipage} &
\begin{minipage}[t]{0.80\columnwidth}\raggedright\strut
Geschickt, Besonnen, Bullenpeitsche einsetzen, Höhenangst
\strut\end{minipage}\tabularnewline
\begin{minipage}[t]{0.14\columnwidth}\raggedleft\strut
\textbf{Ausrüstung}
\strut\end{minipage} &
\begin{minipage}[t]{0.80\columnwidth}\raggedright\strut
Bullenpeitsche, Abgetragene Lederjacke
\strut\end{minipage}\tabularnewline
\begin{minipage}[t]{0.14\columnwidth}\raggedleft\strut
\textbf{Motiv}
\strut\end{minipage} &
\begin{minipage}[t]{0.80\columnwidth}\raggedright\strut
Den Götzen von Toth finden
\strut\end{minipage}\tabularnewline
\begin{minipage}[t]{0.14\columnwidth}\raggedleft\strut
\textbf{Zustände}
\strut\end{minipage} &
\begin{minipage}[t]{0.80\columnwidth}\raggedright\strut
□ Wütend, □ Verängstigt, □ Müde, □ Gefangen, □ Geblendet, □ Hungrig, □
Benommen, □ Verletzt, □ Dem Tode nah
\strut\end{minipage}\tabularnewline
\bottomrule
\end{longtable}

\begin{quote}
Tennessee Smith ist ein friedfertiger Geschichtsprofessor, der in seiner
Freizeit allerdings oft in die Wildnis reist, um verschollene Artefakte
und Schätze zu suchen. Er ist ein kerniger, attraktiver Typ und scheint
immer die Ruhe zu bewahren, egal in welcher Gefahr er sich auch
befindet. Und in Gefahr gerät er ziemlich oft!
\end{quote}

\textbf{Beziehungen:}

\textbf{Hinweise:}

\subsubsection{Jimmy Sweet, Beherzter
Gassenjunge}\label{jimmy-sweet-beherzter-gassenjunge}

\begin{longtable}[]{@{}rl@{}}
\toprule
\begin{minipage}[b]{0.14\columnwidth}\raggedleft\strut
Figur
\strut\end{minipage} &
\begin{minipage}[b]{0.80\columnwidth}\raggedright\strut
Jimmy Sweet
\strut\end{minipage}\tabularnewline
\midrule
\endhead
\begin{minipage}[t]{0.14\columnwidth}\raggedleft\strut
\textbf{Beschreiber}
\strut\end{minipage} &
\begin{minipage}[t]{0.80\columnwidth}\raggedright\strut
Flink, Neunmalklug, Unterschätzt, Jung
\strut\end{minipage}\tabularnewline
\begin{minipage}[t]{0.14\columnwidth}\raggedleft\strut
\textbf{Ausrüstung}
\strut\end{minipage} &
\begin{minipage}[t]{0.80\columnwidth}\raggedright\strut
Laute Feuerwerkskörper, Schmutzige Baseballmütze
\strut\end{minipage}\tabularnewline
\begin{minipage}[t]{0.14\columnwidth}\raggedleft\strut
\textbf{Motiv}
\strut\end{minipage} &
\begin{minipage}[t]{0.80\columnwidth}\raggedright\strut
Die sieben Weltwunder sehen
\strut\end{minipage}\tabularnewline
\begin{minipage}[t]{0.14\columnwidth}\raggedleft\strut
\textbf{Zustände}
\strut\end{minipage} &
\begin{minipage}[t]{0.80\columnwidth}\raggedright\strut
□ Wütend, □ Verängstigt, □ Müde, □ Gefangen, □ Geblendet, □ Hungrig, □
Benommen, □ Verletzt, □ Dem Tode nah
\strut\end{minipage}\tabularnewline
\bottomrule
\end{longtable}

\begin{quote}
Jimmy Sweet ist ein unbekümmerter Gassenjunge, der schon immer auf sich
allein gestellt war. Er ist drahtig, zäh und aufgeweckt, und seine
vorlaute Art bringt ihn regelmäßig in Schwierigkeiten. Jimmy hat sich
als blinder Passagier auf einem Frachtdampfer versteckt, um die Welt zu
sehen und reich zu werden.
\end{quote}

\textbf{Beziehungen:}

\textbf{Hinweise:}

\subsubsection{Harvey Reed, Boxer im
Ruhestand}\label{harvey-reed-boxer-im-ruhestand}

\begin{longtable}[]{@{}rl@{}}
\toprule
\begin{minipage}[b]{0.14\columnwidth}\raggedleft\strut
Figur
\strut\end{minipage} &
\begin{minipage}[b]{0.80\columnwidth}\raggedright\strut
Harvey Reed
\strut\end{minipage}\tabularnewline
\midrule
\endhead
\begin{minipage}[t]{0.14\columnwidth}\raggedleft\strut
\textbf{Beschreiber}
\strut\end{minipage} &
\begin{minipage}[t]{0.80\columnwidth}\raggedright\strut
Stark, Kann schnell improvisieren, Boxer, Hässlich wie die Nacht
\strut\end{minipage}\tabularnewline
\begin{minipage}[t]{0.14\columnwidth}\raggedleft\strut
\textbf{Ausrüstung}
\strut\end{minipage} &
\begin{minipage}[t]{0.80\columnwidth}\raggedright\strut
Münzrolle, Schlecht sitzender Anzug
\strut\end{minipage}\tabularnewline
\begin{minipage}[t]{0.14\columnwidth}\raggedleft\strut
\textbf{Motiv}
\strut\end{minipage} &
\begin{minipage}[t]{0.80\columnwidth}\raggedright\strut
Auf Tennessee Smith aufpassen
\strut\end{minipage}\tabularnewline
\begin{minipage}[t]{0.14\columnwidth}\raggedleft\strut
\textbf{Zustände}
\strut\end{minipage} &
\begin{minipage}[t]{0.80\columnwidth}\raggedright\strut
□ Wütend, □ Verängstigt, □ Müde, □ Gefangen, □ Geblendet, □ Hungrig, □
Benommen, □ Verletzt, □ Dem Tode nah
\strut\end{minipage}\tabularnewline
\bottomrule
\end{longtable}

\begin{quote}
Harvey Reed ist ein frisch pensionierter Boxchampion. Da ihm das
sesshafte Leben ein bisschen zu langweilig ist, hat er sich seinem
Freund Tennessee Smith angeschlossen, um das Abenteuer zu suchen.
\end{quote}

\textbf{Beziehungen:}

\textbf{Bemerkungen:}

\subsubsection{October Jones, Junge
Reporterin}\label{october-jones-junge-reporterin}

\begin{longtable}[]{@{}rl@{}}
\toprule
\begin{minipage}[b]{0.14\columnwidth}\raggedleft\strut
Figur
\strut\end{minipage} &
\begin{minipage}[b]{0.80\columnwidth}\raggedright\strut
October Jones
\strut\end{minipage}\tabularnewline
\midrule
\endhead
\begin{minipage}[t]{0.14\columnwidth}\raggedleft\strut
\textbf{Beschreiber}
\strut\end{minipage} &
\begin{minipage}[t]{0.80\columnwidth}\raggedright\strut
Wunderschön, Geistreich, Gutes Gedächtnis, Neugierig
\strut\end{minipage}\tabularnewline
\begin{minipage}[t]{0.14\columnwidth}\raggedleft\strut
\textbf{Ausrüstung}
\strut\end{minipage} &
\begin{minipage}[t]{0.80\columnwidth}\raggedright\strut
Verlässliche Kamera, Dicker Notizblock
\strut\end{minipage}\tabularnewline
\begin{minipage}[t]{0.14\columnwidth}\raggedleft\strut
\textbf{Motiv}
\strut\end{minipage} &
\begin{minipage}[t]{0.80\columnwidth}\raggedright\strut
Den Exklusivbericht über den wahren Tennessee Smith kriegen
\strut\end{minipage}\tabularnewline
\begin{minipage}[t]{0.14\columnwidth}\raggedleft\strut
\textbf{Zustände}
\strut\end{minipage} &
\begin{minipage}[t]{0.80\columnwidth}\raggedright\strut
□ Wütend, □ Verängstigt, □ Müde, □ Gefangen, □ Geblendet, □ Hungrig,
Benommen, □ Verletzt, □ Dem Tode nah
\strut\end{minipage}\tabularnewline
\bottomrule
\end{longtable}

\begin{quote}
October Jones ist Reporterin und Abenteurerin. Der große Reichtum ihres
Vaters erlaubt ihr viele Freiheiten. Als schöne, intelligente und
verwöhnte Frau ist sie es gewohnt zu bekommen, was sie will.
\end{quote}

\textbf{Beziehungen:}

\textbf{Hinweise:}

\section{Anhang -- Beschreiber}\label{anhang-beschreiber}

Diese Liste ist bei weitem nicht vollständig, aber ein guter Startpunkt.
Jeder Eintrag beschreibt Dinge, für die der entsprechende Beschreiber
vor- oder nachteilig sein könnte. Damit ist sie ein praktisches
Nachschlagwerk sowohl für Erzählerinnen als auch für Spielerinnen.

\subsection{Körperliche Beschreiber}\label{kuxf6rperliche-beschreiber}

\textbf{Beweglich:} hilfreich beim Tanzen, Springen, Balancieren,
Turnen, und Kriechen durch enge Stellen.

\textbf{Beidhändig:} toll beim gleichzeitigen Schießen mit zwei Waffen
und bei Taschenspielertricks.

\textbf{Blondine:} gut, um von Leuten unterschätzt zu werden, nicht gut,
wenn man sich zu lange in der Sonne aufhält.

\textbf{Selbstbräuner-Orange:} toll, um als alternde Berühmtheit
durchzugehen und um Aufmerksamkeit auf sich zu ziehen.

\textbf{Pelzig:} nützlich, um kalter Witterung zu widerstehen und in der
Wildnis zu überleben.

\textbf{Gutaussehend:} nützlich beim Flirten, Dinge verkaufen,
Filmrollen bekommen, Modeln, und fürs beliebt sein.

\begin{itemize}
\tightlist
\item
  \textbf{Gewaltig:} gut, um gefährlich zu wirken, an hohe Dinge
  heranzukommen, eine Bodybuilderin zu spielen oder an engen Stellen
  steckenzubleiben.
\end{itemize}

\textbf{Übergewichtig:} ein Problem beim Sport und beim Ausleihen von
Kleidung.

\textbf{Schwache Konstitution:} hinderlich bei Langstreckenläufen, dem
Aushalten von Schaden, Ausdauertätigkeiten und beim Verheilen von
Wunden.

\textbf{Flink:} nützlich beim Wegducken und von Deckung zu Deckung
rennen, bei Taschenspielertricks, beim Ausweichen und bei anderen
Dingen, die schnelles Handeln erfordern.

\textbf{Rasiermesserscharfe Klauen:} großartig, um Gegner
aufzuschlitzen, Obst und Gemüse zu schneiden und eventuell auch beim
Bäumehochklettern

\textbf{Klein:} nervig, weil man nicht an hohe Sachen herankommt, aber
praktisch, um unter niedrige Dinge zu kriechen und in der Menge
unterzutauchen.

\textbf{Langsam:} ein Problem beim Rennen, Ausweichen und Reagieren.

\textbf{Stark:} sinnvoll, um Dinge anzuheben, zu tragen, zu zerschlagen
und zu werfen; könnte auch beim Ringen helfen und dabei, nicht von
Dingen erdrückt zu werden.

\textbf{Hochgewachsen:} nützlich, um an das oberste Regal heranzukommen,
beim Klettern und um über die Köpfe anderer Leute hinwegzuschauen.

\textbf{Dünn:} praktisch, um sich in Kleidung und enge Räume
hineinzuzwängen, sich hinter Pfosten zu verstecken und auf dem Laufsteg
eine gute Figur zu machen.

\textbf{Hässlich:} ein Problem bei der Verführung oder wenn man eine
Fernsehrolle haben will.

\textbf{Vitalität:} großartig, um der Wirkung von Giften zu widerstehen,
bei Langstreckenläufen und anderen Ausdauertätigkeiten.

\textbf{Schwächlich:} lästig beim Anheben, Tragen, Zerschlagen und
Werfen von Dingen

\textbf{Schwimmfüße:} toll fürs Schwimmen, aber schrecklich, wenn man
Schuhe kaufen will.

\subsubsection{Geistige Beschreiber}\label{geistige-beschreiber}

\textbf{Zerstreut:} gut, um sich abzulenken, aber problematisch, wenn
man nicht vergessen will, wo man seine Schlüssel gelassen oder dass man
gerade den Sicherheitsstift einer Handgranate gezogen hat.

\textbf{Belesen:} toll, um Schulprüfungen abzulegen, Matheformeln zu
kennen, sich an Geschichtsdaten zu erinnern und um die Zulassung zum
Jurastudium zu erlangen.

\textbf{Computerfreak:} gut beim Programmieren und Laptopreparieren, und
um in Internet-Kleinkriege zwischen PC- und Apple-Fans zu geraten.

\textbf{Dämlich:} problematisch, wenn man Witze verstehen, Tricks
bemerken oder ganz allgemein in Gesellschaft den Schein wahren will.

\textbf{Einfühlsam:} gut, um die Gefühle anderer Menschen zu lesen, um
psychologische Diagnosen zu stellen und bekümmerte Menschen zu trösten.

\textbf{Konzentriert:} hilfreich um bei der Sache zu bleiben, sich nicht
ablenken zu lassen und um ernst dreinzuschauen.

\textbf{Querdenkerin:} nützlich, um Probleme zu lösen und
Schwierigkeiten kreativ anzugehen.

\textbf{Mathematikerin:} toll, um zu addieren, Gleichungen zu lösen und
die Steuererklärung machen.

\textbf{Aufmerksam:} prima, um versteckte Hinweise und Details zu
erkennen, Wortsuchrätsel zu lösen und beim Lesen von Körpersprache.

\textbf{Rednerin:} vorteilhaft beim Halten von öffentlichen Reden,
Debatten führen und Klarmachen deines Standpunkts.

\textbf{Langsam:} nervig, wenn man Pläne verstehen oder neue Dinge
lernen will.

\textbf{Taktikerin:} großartig, um Schlachten zu planen, sich an
Militärgeschichte zu erinnern und Sun-Tsu zu zitieren.

\textbf{Ungebildet:} nervig beim Lesen, Lösen von Matheaufgaben, Merken
wichtiger Geschichtsdaten und wenn man Schulprüfungen bestehen will

\textbf{Weise:} nützlich, um Sprichwörter zu zitieren, ratzugeben,
Hinweise in Verbindung zu bringen, Reaktionen anderer Menschen zu
interpretieren und anzumerken: ``Ich hab's euch ja gesagt''.

\textbf{Geistreich:} gut, um witzige Kommentare rauszuhauen,
charmant/unterhaltsam zu sein, das Richtige zum rechten Zeitpunkt zu
sagen.

\subsection{Beschreiber für
Stärken}\label{beschreiber-fuxfcr-stuxe4rken}

\textbf{Akrobatik:} großartig, um zum Zirkus zu gehen, durch schmale
Öffnungen zu springen und beeindruckende Salti zu schlagen.

\textbf{Okkultes Geheimwissen:} praktisch, um mystische Artefakte zu
identifizieren, schwarze Magie zu erkennen und uralte Pergamentrollen zu
lesen.

\textbf{Mut:} hilreich, um einen grauenvollen Anblick zu ertragen; deine
Frau zu informieren, dass du den Hochzeitsstag vergessen hast, und um
andere gefährliche Aktionen zu versuchen.

\textbf{Fahren:} toll bei Autorennen, Verfolgungsjagden und beim
Bestehen deiner Fahrprüfung

\textbf{Fechten:} gut für den Schwertkampf und andere zivilisierte
Formen des Nahkampfes.

\textbf{Gutes Gedächtnis:} nützlich, um sich an Namen und Gesichter,
wichtige Hinweise und mathematische Formeln zu erinneren.

\textbf{Hunting:} good when tracking and stalking, look- ing good in
camou age, and knowing what an angry rhinoceros sounds like.
\textbf{Jagen:} vorteilhaft beim Spurenlesen und Anpirschen, um in
Tarnkleidung gut auszusehen, und um zu wissen, wie sich ein wütendes
Nashorn anhört.

\textbf{Scharfe Augen:} super, um weit entfernte Dinge sehen oder sogar
im Mondlicht arbeiten zu können.

\textbf{Sprachwissenschaften:} Vorteile beim Sprechen von Fremdsprachen
und allgemein bei der Kommunikation mit anderen Leuten.

\textbf{Magie:} Vorteile beim Beherrschen mystischer Künste, beim Wirken
von Zaubersprüchen oder beim Auftritt als Bühnenmagier.

\textbf{Medizin:} hilfreich beim Durchführen von Operationen,
Diagnostizieren von Krankheiten, und Leisten von Erster Hilfe.

\textbf{Fieser Biss:} super, um Gegner im Kampf schwer zu verletzen,
sich den eigenen Arm abzukauen oder einen Kuchenesswettbewerb zu
gewinnen.

\textbf{Reich:} praktisch, um Luxussportwagen zu kaufen, Einladungen zu
exklusiven Partys zu bekommen und um städtische Beamte zu bestechen.

\textbf{Ringkampf:} gut beim unbewaffneten Kampf und dem Festsetzen von
Gegnern am Boden.

\subsection{Beschreiber für
Schächen}\label{beschreiber-fuxfcr-schuxe4chen}

\textbf{Blind:} nervig bei allem, was mit dem Sehvermögen zu tun hat,
zum Beispiel Schießen, sich auf unbekanntem Gelände Zurechtfinden oder
Malen

\textbf{Mutig:} nützlich beim Heranstürmen in eine tödliche Gefahr, bei
tollkühnem Verhalten und um ordentlich in Schwierigkeiten zu geraten.

\textbf{Ungeschickt:} problematisch beim Tragen einer wertvollen Vase,
beim Besuch eines Antiquitätenladens oder dem Durchqueren eines mit
Fallen gespickten Raumes

\textbf{Gierig:} ungünstig beim Widerstehen des Impulses, etwas zu
stehlen, zu lügen oder anderweitig Reichtümer anzuhäufen oder zu
behalten.

\textbf{Unmenschliches Aussehen:} hinderlich, wenn man nicht bemerkt
werden möchte, Aufmerksamkeit vermeiden will, oder perfekt sitzende
Hosen sucht.

\textbf{Fehlendes Bein:} Problematisch beim Laufen, Klettern oder jeder
anderen Aktivität, die Bewegung ohne Prothesen oder Rollstuhl umfasst.

\textbf{Alt:} Probleme wenn man cool aussehen, Treppen steigen oder
Computer benutzen will oder zuversichtlich zu sein, wenn es um die
eigene Gesundheit geht.

\textbf{Arm:} hinderlich, wenn man Essen oder Kleidung kaufen oder an
einer exklusiven Party teilnehmen will.

\textbf{Sieht schlecht:} nervig, wenn man jemanden oder etwas
wiedererkennen will, beim Fahren im Dunkeln oder beim Ausmachen
visueller Hinweise.

\textbf{Primitiv:} Probleme beim Nutzen von Handys, Autos oder
Türklingeln und beim Benehmen auf kultivierten gesellschaftlichen
Anlässen.

\textbf{Intensiver Geruch:} hinderlich, wenn man Leute beeindrucken oder
sich vor wilden Tieren und Verfolgern verstecken will.

\textbf{Gesucht:} nervig, wenn man sich von Schwierigkeiten fernhalten
will oder man etwas aus seiner Wohnung braucht.

\textbf{Jung:} problematisch, wenn man in Diskos rein möchte, von
Erwachsenen ernstgenommen werden will, die Schulpflicht umgehen oder im
Auto über das Armaturenbrett hinausblicken möchte.

\subsection{Figurenvorlage}\label{figurenvorlage}

Eine ausdruckbare Figurenvorlage zum Ausfüllen ist online unter
folgender Adresse zum Herunterladen erhältlich:

(Dies ist der Link zum franz. Bogen, wir sollten den deutschen ins Git
einpflegen und dann hier verlinken)

\url{https://raw.githubusercontent.com/brunobord/fu-rpg-libre-et-universel/master/feuille-de-personnage.pdf}

\end{document}
